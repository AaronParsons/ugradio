\documentclass[preprint]{aastex}
\usepackage{graphicx}

%\documentclass[11pt]{article}
%\textwidth 7.0in 
%\hoffset=-0.9in
%\textheight 9.5in 
%\voffset=-1.0in 
%\parindent=.5in 

\begin{document}
\tolerance=10000

%\parskip=0.2cm
%\parindent=1.0cm
\def\simlt{\lower.5ex\hbox{$\; \buildrel < \over \sim \;$}}
\def\simgt{\lower.5ex\hbox{$\; \buildrel > \over \sim \;$}}


\title  {TRANSMISSION LINES, WAVEGUIDES, IMPEDANCE MATCHING}

How do you move power or electronic information from one place to
another?\footnote{Yeah, there's wireless these days. But bit rates are
  limited because bandwidths are limited (remember the Nyquist criterion!)
  and\dots try powering the city of Berkeley using wireless!} Transmission
lines (cables) and waveguides are indispensible. Sounds easy---just connect
two things with a wire. But does this really work? We'll explore cables,
waveguides, and reflections by measuring the Voltage Standing Wave
Ratio (VSWR), which results from interference of incident and reflected
waves.

\section{GOALS}

\begin{itemize}

\item Learn about reflections and minimizing them with impedence
  matching. 

\item Derive propagation velocities by measuring the wavelength of
  standing waves.

\item Measure cable lengths by measuring reflections. 

\item Empirically explore waveguides and their cutoff properties.

\end{itemize}

	The experimental work and measurments described in \S
\ref{expt} should be done by groups.  The analysis in \S
\ref{secondweek} should be done by individuals. The analysis is
nontrivial, so don't delay with the measurements!

\section {VSWR MEASUREMENTS WITH SLOTTED LINES
AND WAVEGUIDES} \label{expt}

{\boldmath
\subsection {The coax slotted line at C-band ($\sim 3$ GHz)} \label{slotted}
}
	Set up the slotted coaxial line with the C-band oscillator and
cavity wavemeter.  With the far end of the slotted line open, measure
the wavelength in the cable, measure the positions of {\it all} nulls as
accurately as you can.  Do the same with the far end shorted and note
the differences.  In particular, note how the positions of the nulls
change when the slotted line is terminated with an open versus when it
is shorted.  Why is this?

	From your measurements calculate the velocity [what does
``velocity'' mean?] in the slotted line by comparing the wavelength and
the frequency.  Do this as accurately as possible by following the
advice in \S \ref{slottedls}.  {\it Note:} As it happens, the velocity
is independent of frequency for a coax cable---in contrast to a
waveguide.  If you are so motivated, you can check this experimentally. 

	Put a matched load, and then a calibrated mismatched load, on
the end of the slotted line and measure the VSWR using a few
null/maximum pairs.  Also use an ohmmeter to measure the DC resistances
of the two loads, and determine whether your results agree with the
theory in \S \ref{VSWR}; while you're at it, measure the DC resistance
of the cable and ponder why it is called ``50-ohm cable''.

\subsection{The flexible coax at C-band} 

	Attach a reasonably long piece of coax cable to the far end of
the slotted line and, using the slotted line, measure the wavelength
(and therefore velocity) {\it in the flexible coax.} It is easy to
measure the wavelength in the {\it slotted} coax, but it requires real
insight and cleverness to determine the wavelength in the {\it flexible}
coax. The yellow coax is a good choice; previous classes measured its
length as 855 cm (I think).

	{\it Hint:} To make this measurement, you need to change a
parameter and observe the results of the change. What parameters can you
change? Which one allows you to do the job? ({\it HEY!!!} You're {\it
NOT ALLOWED} to change the length of the cable by cutting it!!!!)

\subsection {The X-band waveguide (7 to 12 GHz)} \label{waveguide}

	Now we'll do the same as we did in \S \ref{slotted}, but for
X-band waveguide.  With the far end of the waveguide completely open,
measure the VSWR; then short the end of the waveguide, maybe with a
piece of aluminum foil or a metal plate, and repeat.  Why is an open
ended waveguide different from an open ended coax cable?

	With the end shorted, measure the velocity in the slotted
waveguide by comparing the wavelength and the frequency; do this for
several frequencies and compare with theory.  These frequencies should
span the full range available and be reasonably closely separated, say
by no more than 1 GHz.  From these measurements, derive the cutoff
wavelength (and thus the waveguide width) as accurately as you can. 
Also measure the waveguide dimensions with a caliper, and compare this
with the width you derived above.  Finally, play around with the cutoff
frequency behavior and, if you have time, see how the transmission
behaves in the vicinity of cutoff. 

\section {GETTING THE MOST OUT OF YOUR DATA: STATISTICAL ERROR
ANALYSIS} \label{secondweek}

\subsection{The coax slotted-line wavelength} \label{lbandcoax} \label{slottedls}

	For the measurements of \S \ref{slotted}, you have sampled $M$
cycles of the standing wave. You can calculate the wavelength from the
distances between a single null pair. While this calculation is a good
estimate, each of your measurements has an error. You have measured the
distance between $(M-1)$ null pairs. What's the best way to combine
these measurements so as to obtain the most accurate wavelength?

	{\bf (1)} One way is to calculate the wavelength from each
neighboring null {\it pair} and take the average of the $(M-1)$ {\it
pair}s.  You might then apply the usual rule to obtain the uncertainty
in the average, i.e.\ the uncertainty is $(standard \ deviation) \over
(M-1)$.  However, this would {\it not be correct}.  Why? (Hint: The
answer you get by averaging the $(M-1)$ pairs is identical to that
obtained from a {\it single} well-chosen non-neighbor pair.  Which
one?).  Actually, this method gives {\it almost} the best estimate of
the wavelength.  But you can do better!

	{\bf (2)} The positions of the nulls should increase linearly
with distance. So if $x_{m}$ is the position of null number $m$, we
should have

\begin{equation}
x_m = A + m {\lambda \over 2} \ ,
\end{equation}

\noindent You know $x_m$ and $m$ from measurement and you would like to
derive $A$ and $\lambda$. This is a classic least squares problem. In
fact, it is one of the simplest because it is the lowest-order polynomial
fit. You don't have to write your own polynomial fit; instead, you can
use IDL's \verb$poly_fit$ or my \verb$polyfit$. 

\subsection{The X-band waveguide}

Here we want to do a least-squares fit of equation \ref{lambdaguide} to
solve for the waveguide width $a$. Now, you've already measured $a$ with
the caliper. But here the idea is to compare your measurements of the
waveguide cutoff with what theory (that's equation \ref{lambdaguide})
predicts. So you imagine that $a$ is unknown and determine it from your
data on $\lambda_g$ versus $\lambda$.  Doing this is not as straightforward
as above in section \ref{lbandcoax} because $\lambda_g$ depends {\it
  nonlinearly} on $a$. From previous labs you know how to approach this
problem. Show enough of your work so that I know what you've done.

	From your above solution, determine the best-fit value of $a$
and compare it with the value you measured with the caliper.

\section{THEORETICAL COMMENTARY---A MUST READ!}

	The material in this section is essential for your work in this
lab.  It is extracted from the excellent book by Ramo, Whinnery, and van
Duzer (RWvD).  If you are so inclined, take a look at that book; we
provide commentary on the book in \S \ref{RWvD}. 

\subsection{The Voltage Standing Wave Ratio (VSWR)} \label{VSWR}

	The voltage reflection coefficient for a load with resistance
$R_L$ that terminates a line with impedance $Z_0$ is

\begin{equation}
\rho = {V_R \over V_F} = {R_L - Z_0 \over R_L + Z_0}
\end{equation}

\noindent where $V_R$ is the reflected voltage amplitude and $V_F$ the
forward, or incident, one. The reflected voltage interferes
constructively with the forward one, giving a maximum $V_{max}=
V_{forward} + V_{reflected}$; and destructively, giving a minimum
$V_{min}= V_{forward} - V_{reflected}$.  

	In more detail, we can write

%\begin{mathletters}
\begin{equation} 
V_{forward} = V_F e^{i2 \pi (f t - x/\lambda)} \ ,
\end{equation}

\noindent which travels towards positive $x$, and 

\begin{equation} 
V_{reflected} = V_R e^{i 2 \pi (f t + x/\lambda)} \ ,
\end{equation}
%\end{mathletters}

\noindent which goes the other direction. These add to give

%\begin{mathletters}
\begin{equation}
V_x = V_F e^{i2 \pi f t} 
	\left[ e^{-i 2 \pi x/\lambda} + \rho e^{i 2 \pi x/\lambda} \right])
\end{equation}

\noindent which can be written as

\begin{equation} V_x = V_F e^{i2 \pi f t} \tolerance=10000\left\{
 {1 + \rho \over 2} \left[ e^{i 2 \pi x/\lambda} + e^{-i 2 \pi x/\lambda} \right]
-{1 - \rho \over 2} \left[ e^{i 2 \pi x/\lambda} - e^{-i 2 \pi
x/\lambda} \right]
\right\}
\end{equation}

\noindent or, in the conventional trig notation that we all understand,
\begin{equation} 
V_x = V_F e^{i2 \pi f t} \left\{ 
 (1 + \rho) \cos( 2 \pi x/\lambda)
-(1 - \rho) \sin( 2 \pi x/\lambda) \right\}
\end{equation}

\noindent So the standing wave pattern repeats every wavelength with an
amplitude and phase (position) that depend on $\rho$. With our
slotted-line probe, we measure the
mean {\it absolute value} of the internal electric field, which is proportional to
$|V_x|$. As we move along in $x$, $|V_x|$ oscillates sinusoidally beween
values $|V_{max}|$ and $|V_{min}|$ with
period $\lambda$. The VSWR is the max-to-min ratio (see RWvD \S 8.8,
equation 11)

\begin{equation} \label{vswreqn}
VSWR = {|V_{max}| \over |V_{min}|} = {1 + \rho \over 1 - \rho}
\end{equation}

\subsection{The $TE_{10}$ Mode in a Rectangular Waveguide.} 

	RWvD provide a beautiful discussion of the TE$_{10}$ mode in
common rectangular waveguide (in which the ratio of side lengths equals
2).  Most important is their equation 11, which gives the ``guide
wavelength'' $\lambda_g$; we reproduce it here:

\begin{equation} \label{lambdaguide}
\lambda_g = {v_p \over f} = {\lambda \over \left[ 1 - (\lambda/2a)^2
\right]^{1/2} } \ ,
\end{equation}

\noindent where $\lambda$ is the free-space wavelength.  The guide
wavelength is the wavelength within the waveguide, and combining it with
the frequency gives the propagation velocity in the guide. 

\section {REFERENCE READING} \label{RWvD}

You can do this lab without doing any reading\dots but then you wouldn't
gain the insight from the highly readable and well-written text
by Ramo, Whinnery, and van Duzer (RWvD), {\it Fields and Waves in
Communication Electronics}. This is a great book because it contains
commentary on applications to modern devices (e.g.,  the discussions of
transmission lines carrying pulse from one piece of computer hardware to
another are very nice).  In the course reference binder, you can find
copies of the following sections, for which we provide commentary below. 
The binder and notes below refer to the second edition. 

	If you have the time and inclination to read, we suggest that
you concentrate on the starred items below ({\bf ***}).  The italicized
numbers refer to the very same sections of the book.


\subsection{ Theory of Coaxial Cable}
	
{\bf ***} {\it RWvD \S 5.2:  Introduction: what transmission lines do.}
Basic equations for transmission lines.  Note
equation (14) for the characteristic impedance $Z_0$, which is the ratio
of voltage to current in the line.  This is expressed in ohms, but it is
not ``ordinary resistance'' because voltage and current are out of
phase.  Example 5.2 calculates $Z_0$ for a coax line.  RG58/U, done in
the example, has 50 ohms; so do the cables we use in our system (in
contrast, the TV standard is 75 ohms).  Because of the dielectric in the
cable, the wave (phase) velocity is $\sim 0.7$ that in free space,
making the wavelength in the cable shorter than that in free space; one
needs to account for this when using cables to produce delays in a
signal, as in an interferometer or a quadrature circuit. 

{\bf ***} {\it RWvD \S 5.4: Reflection and transmission at a resistive
discontinuity.} The purpose of a transmission line is to take power from
a source to a load. The load should absorb the power and therefore
should be resistive. If its resistance matches the line impedance, then
all the power transmitted by the line is absorbed by the load. If not,
some is reflected; see equations (4) and (6). We need to tune our
devices so that they are ``matched'' and reflect no power. 

          {\it RWvD \S 5.6: Standing Wave Ratio.} Finally: how do we
measure mismatch? RWvD's example 5.6 discusses the measurement of
slotted line impedance measurements, which we will do in the lab.  (This
only appears in the Second Edition photocopy of RWvD.) With the slotted
line, the probe samples the electric field within the cable, which is
proportional to the voltage.  Thus we measure the ratio $V_{max} \over
V_{min}$ in equation \ref{vswreqn}, from which we calculate
$VSWR$---which is related to the power reflected in RWvD's equation (6). 

\subsection{The Theory of Waveguide}
          
{\bf ***} {\it RWvD \S 8.1:  Introduction: general description of
waveguides.} 

{\bf ***} {\it RWvD \S 8.8:  The $TE_{10}$ Mode in a Rectangular
Guide.} A nice, complete discussion of the TE$_{10}$ mode in common
rectangular waveguide (in which the ratio of side lengths equals 2). 
Most important is equation (11) (same as our equation \ref{lambdaguide}
above), which gives the ``guide wavelength'' $\lambda_g$. 

          In this mode, E is perpendicular to the long side of the guide
(and also to the longitudinal axis of the guide).  We speak of the
polarization of the waveguide: the guide transmits linear polarization
that is perpendicular to the long side of a waveguide.  When we couple
power into a guide from coax, we do so with a probe that is parallel to
the internal electric field.  When we use a sliding probe to measure
VSWR in a waveguide, the probe extends through a long slot in the broad
side of the waveguide and the probe is parallel to the E field inside
the guide, so it samples the internal E field.  Note {\bf Figure 8.8},
which shows the current flow in the waveguide walls; note that the
current flow down the center of the broad side of the guide is parallel
to the axis of the guide---that is, parallel to the slot that we use for
the probe.  The slot, being parallel to the current flow, does not
interrupt any current flow and has no effect on the internal
distribution of fields in the guide.  Clever!

          {\it RWvD \S 8.10.} Makes the analogy between coax cable and
waveguide.  A coaxial line can be considered as a waveguide containing
walls not only the outside but also on the inside.  The usual mode that
propagates in coax is the TEM mode.  A TE$_{10}$-like waveguide mode
enters for wavelengths smaller than that given by equation (4); at this
point, coax lines become basically unusable.  For type N connectors, we
are beginning to approach this limit in our 12 GHz system!

{\it RWvD \S 8.11.} Brief description of how coax lines are coupled to
waveguide. The method depends on which mode you wish to excite. For the
TE$_{10}$ mode, method (b) in Figure 8.11 works well and is almost
universally the one used.

\subsection {Theory of the Directional Coupler}

          Section 11.11 at the beginning (page 551, 552), and Figure
11.11, make it clear how these devices work. The clever use of phase to
make a signal interfere constructively and/or destructively with itself
is basic to many devices, some of which we will use.

\end{document}

