\documentclass[preprint]{aastex}

\begin{document}
\setlength{\parskip}{0.02in}

\title{A Beginners Guide to Emacs}

\author{Nicholas Szandor Hakobian; updated by Carl Heiles}
\affil{University of California, Berkeley, Undergraduate Radio
Astronomy Laboratory}
\affil{Version 2, \today}
\email{nick@ugastro.berkeley.edu}

{\bf Jan 2017: When logged in from within the lab, to make Emacs
  function properly you must enter the following command (including the
  [] ). You only need do it once---not every time you log in):}

{\tt gsettings set org.freedesktop.ibus.general.hotkey triggers []}


\section{So What Is Emacs?}
Emacs is a text file editor in Unix environments. This includes our
lab which runs a variant of Unix called Solaris, and other, more
readily available operating systems such as Linux. You can use Emacs
to quickly edit a simple text file, as an editor for your IDL source
code, or as a nice front end to typesetting documents in \LaTeX, the
required typesetting utility in this class. This document will outline
the basics of using Emacs.

\section{The Basics}

\subsection{The {\tt -nw} Option}
Emacs is normally run in a specially-created X-window. This can make it
slow if you are running on a remote computer, e.g.\ as an astronomer who
is editing a file on a distant telescope's computer system. You can get
around this by running Emacs in the window from which you invoke it; the
formatting is a little less fancy and mouse won't work, so you have to
use the keyboard shortcuts, but the response time is much quicker. To
run Emacs without generating a new window, type {\tt emacs -nw filename}
instead of {\tt emacs filename}; the {\tt -nw} means `no window'.

\subsection{Opening}
Most of the programs that you will be using on these systems will be
started from the command line. To begin a new document, simply
navagate (via the Unix command line) to the directory where you want
the file to be and type: 
\begin{verbatim}
emacs something.txt &
or
emacs -nw something.txt &
\end{verbatim}
Where {\tt something.txt} is the new file that you wish to create, or
the old file that you wish to update. Without the {\tt -nw} option,
after a few moments a new window will popup with a blank screen that you
can type into. You can use the mouse to navigate the menus at the
top. See table 1 for information on what the different menu options do.

\begin{table}[h]
\begin{center}
\begin{tabular}{|c|c|} \hline
Buffers & Contains information about the currently open files in this
instance of \\ & Emacs.\\ \hline
Files & Contains basic commands for opening, saving, and closing
files.\\ \hline
Tools & Contains advanced functions for doing things like version
control.\\ \hline
Edit & Contains cut, paste, and spell checking capabilities of
emacs.\\ \hline
Search & Contains search and replace functions.\\ \hline
Mule & Advanced editing functions.\\ \hline
TeX & LaTeX specific functions. Only appears when the filename you are
\\ & editing ends with \emph{.tex}.\\ \hline
IDL or IDLWAVE & IDL specific functions. Only appears when you are
editing a filename\\ & that end with \emph{.idl} or \emph{.pro}\\ \hline
Help & Emacs' extensive help and configuration area. \\ \hline
\end{tabular}
\end{center}
\caption{Emacs Menu Items}
\end{table}

Now that you have emacs open you can type something in the text
area. In the next section we will save the file.

\subsection{Saving}
In Emacs, saving a document is as simple as $\pi$. You can click on
the \verb1Files1 menu, then click on \verb1Save Buffer...1 You can
also save by using the keyboard shortcut \verb1C-x C-s1. In the
interest of simplifing your life, this document will concentrate on
the usage of keyboard shortcuts in using Emacs. 

\subsection{Keyboard Shortcuts}
Just about every aspect of Emacs can be accessed and configured with
keyboard shortcuts. This means that to work on a document, you rarely
need to move your hands away from the keyboard and use the mouse. This
increases productivity, and just makes things easier. All the keyboard
shortcuts that I mention in this document will be summarized at the
end.

All keyboard shortcuts are represented in a format like
\verb1C-x C-f1. When you see a \verb1-1, that means you hold the key
that comes before and after the \verb1-1 down at the same time. The
capitol \verb1C1 however, does not correspond to the letter c, but
rather the control button (abbreviated \verb1CTRL1) on the
keyboard. So, if you see \verb1C-x C-f1, that means press and hold the
\verb1CTRL1 button, then press the \verb1x1 key, then press the
\verb1f1 key, then release the \verb1CTRL1 button. If you see
something of the form \verb1C-x d1, then that means to press and hold
the \verb1CTRL1 button, press \verb1x1, release \verb1CTRL1, then
press \verb1d1. I think you get the picture.

In some of the menus, you will see keyboard shortcuts that are
labelled as \tt M-\% \normalfont. On the Sun workstations that
corresponds to the $\Diamond$, and works identically to the
\verb1CTRL1 key. On non-Sun workstations this key is simulated by
pressing \verb1ESC1 then \verb1CTRL-1whatever.

Some basic  keystrock {\it Shortcuts}: \\
Opening - \verb1C-x C-f1\\
Saving - \verb1C-x C-s1\\
Exiting Emacs - \verb1C-x C-c1\\
Pasting - \verb1C-y1\\
Deleting a line - \verb1C-k1\\
Move to the beginning of a line - \verb1C-a1\\
Move to the end of a line - \verb1C-e1\\

\subsection{Cutting, Pasting, and Navigation}
In Emacs cutting and pasting is simple, yet slightly different from what
you may be used to. {\it To do it using keystrokes only}:
\begin{enumerate}

\item Select and cut the block of text by \begin{enumerate}
  \item moving the cursor to its beginning, type {\tt C-spacebar}.
  \item using the arrow keys, move the cursor to its end abd type {\tt C-k}.
\end{enumerate}

\item Paste the block by \begin{enumerate}
  \item moving the cursor to where you want to paste
    \item type {\tt C-y} .
\end{enumerate}
\end{enumerate}

{\it To do it using the mouse}: Select the text with your mouse and
the double right click in the selection. To paste, move the cursor to
where you want to paste and click the middle mouse button. 

\section{\LaTeX Integration}
Emacs has nice \LaTeX integration. If you start a file with the ending
\emph{.tex}, Emacs will automatically load up in a mode to highlight
special characters and special modes to make life a little
easier. Emacs also has a couple modes to automatically compile \LaTeX
files and display the result in xdvi (refer to the \LaTeX handout if
none of this makes sense to you. Below are the basic \LaTeX shortcuts.

\begin{table}[h]
\begin{center}
\begin{tabular}{|c|c|} \hline
Compile \LaTeX file & \verb1C-c C-f1\\ \hline
View file in xdvi & \verb1C-c C-v1\\ \hline
\end{tabular}
\end{center}
\caption{\LaTeX - Emacs commands}
\end{table}

\section{IDL Integration}

Emacs also has a nice mode for integrating with IDL. For the purposes
of this lab, we will only need to use the highlighting features of
Emacs. This checks to make sure that you have matching parentheses,
that you correctly close if...then...endif statements, and will color
special keywords separately from other commands. It is recommended
that you execute your IDL files from the IDL command prompt. 


\end{document}
