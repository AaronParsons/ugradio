\documentclass[psfig,preprint]{aastex}

%\documentstyle[11pt,psfig,aaspp4]{article}
\begin{document}
\title{\bf QUICK IDL TUTORIAL NUMBER TWO: IDL DATATYPES AND
ORGANIZATIONAL STRUCTURES\\ \today}

\tableofcontents

	By the term {\it datatype} we mean, for example, integers,
floating point variables, strings, complex numbers.  By the term {\it
organizational structures} we mean scalars, vectors, arrays, and
structures.  We cover these in the following sections.  All
available datatypes can be arranged in all available organizational
structures.  For example, we can have arrays of strings, vectors of
complex numbers. 

\section {DATATYPES}

	Here we cover only the basic IDL datatypes.  There are others,
including unsigned integers and complex numbers. 

\subsection{ DIGITS, BITS, BYTES, AND WORDS}

	We have gotten to the place where you need to know a little
about the internal workings of computers. Specifically, how the computer
stores numbers and characters.

	Humans think of numbers expressed in powers-of-ten, or {\it
decimal numbers}.  This means that there are 10 digits $(0 \rightarrow
9)$ and you begin counting with these digits.  When you reach the
highest number expressible by a single digit, you use two digits and
generate the next series of numbers, $10 \rightarrow 99$.  Let $f$ and
$s$ be the {\it first} (1) and {\it second} (0) digits, respectively; then the
number is $f*10^1 + s*10^0$.  And so on with more digits. 

	Fundamentally, all computer information is stored in the form of
{\it binary numbers}, meaning powers-of-two.  How many digits? Two! They
are 0 and 1.  The highest number expressible by a single digit is 1. 
The two-digit numbers range from 10 to 11; the number is $f*2^1 + s*2^0$.  And
so on with more digits.  But wait a minute! The word ``digit'' is a
misnomer---it implies something about 10 fingers.  Here it's the word
{\bf bit} that counts.  Each binary ``digit'' is really a {\bf bit}.  So
the binary number 1001 is a 4-bit number.  What decimal number does the
binary number 1001 equal?

	For convenience, computers and their programmers group the bits
into groups of eight. Each group of 8 bits is called a {\bf byte}.
Consider, then, the binary number 11111111; it's the maximum-sized
number that can be stored in a byte. What is this number? 

	Finally, computers group the bytes into {\bf words}.  The oldest
PC's dealt with 8-bit words---one byte.  Many present-day PCs use 32-bit
words---four bytes.  Macs and the more capable PCs deal with 64-bit
words---8 bytes. What's the largest number you can store in a 4-byte
word? And how about negative numbers?

	Below we describe how IDL (and everybody else) gets around this
apparent upper limit on numbers.  They do this by defining different
data types.  Up to now, the details didn't matter much.  But now\dots We
don't cover all datatypes below---specifically, we omit Complex (yes,
complex numbers!), Hexadecimal, Octal;
you can look these up if you are interested. 

\subsection {INTEGER DATATYPES IN IDL}

	{\it Integer} datatypes store the numbers just like you'd
expect. IDL supports integers of four different lengths: 1, 2, 4, and 8
bytes. The shorter the word, the less memory required; the longer the
word, the larger the numbers can be. Different requirements require
different compromises.

\subsubsection {1 byte: The Byte Datatype}

	The Byte datatype is a single byte long and always positive. 
Therefore, its values run $0 \rightarrow 255$.  The byte datatype isn't all
that useful for general computation, but it serves two essential purposes
inside your computer: \begin{enumerate}

  \item Images are always
represented in bytes.  The {\it data} might not be in bytes, but the
numbers that the computer sends to the video processor card are always
bytes.  Video screens require lots of memory and really quick processing
speed, so bytes are ideal.  When you write an image to the screen, which is
done with the {\tt tv} command in IDL, you send a 2-d array---and those
numbers must be bytes. Usually your image is calculated as floats (ordinary
numbers with decimal points; see \S \ref{floats} below); convert these to
bytes using IDL's {\tt bytscl} command.

\item Characters (like the string variable `a') are represented as
  bytes. This means that computers only have 255 chacters that can be used
  in string variables (see \S \ref{strings} below).
\end{enumerate}

You can generate a byte array using {\bf bindgen};
you can generate a single byte variable by saying {\bf
x=3b}.  If, during a calculation, a byte number exceeds 255, then it
will ``wrap around''; for example, 256 wraps to 0, 257 to 1, etc. 

\subsubsection{2 bytes: Integers and Unsigned Integers} 

	With 2 bytes, numbers that are always positive are called {\bf
Unsigned Integers}.  They can range from $0 \rightarrow 256^2 -1$, or $0
\rightarrow 65535$.  You generate an array using {\bf uindgen}.  How do
you think unsigned integers wrap around?

	Normally you want the possibility of negative numbers and you
use {\bf Integers}.  The total number of integer values is $256^2 =
65336 \ (=2 \cdot 32768)$.  One possible value is, of course, zero.  So the number of
negative and positive values differ by one.  The choice is to favor
negative numbers, so Integers cover the range $-32768 \rightarrow
32767$.  You generate an array using {\bf indgen}.  What happens with
wraparound? What if {\bf x=5}, {\bf y=30000} and {\bf z=x*y}? Check it
out!

\subsubsection{4 bytes: Long Integers and Unsigned Long Integers} 

	The discussion here is exactly like that for 2-byte integers,
except that $256^2$ becomes $256^4$.  What are the limits on these
numbers? See IDL help under ``Data Types'' and ``Integer Constants'' for
more information.  You generate arrays using {\bf ulindgen} and {\bf
lindgen}. 

\subsubsection{8 bytes: 64-bit Long Integers and Unsigned 64-bit Long
Integers}

	The discussion here is exactly like that for 2-byte integers,
except that $256^2$ becomes $256^8$.  What are the limits on these
numbers? See IDL help under ``Data Types'' and ``Integer Constants'' for
more information.  You generate arrays using {\bf ul64indgen} and {\bf
l64indgen}. 

\subsection {FLOATING DATATYPES IN IDL} \label{floats}

	The problem with integer datatypes is that you can't represent
anything other than integral numbers---no fractions! Moreover, if you
divide two integer numbers and the result should fractional, but it
won't be; instead, it will be rounded down (e.g.  $5 \over 3$ is
calculated as 1).  To get around this, the {\it floating} datatype uses
some of the bits to store an {\it exponent}, which may be positive or
negative.  You throw away some of the precision of the integer
representation in favor of being able to represent a much wider range of
numbers. 

\subsubsection{4 bytes: Floats}

	``Floating point'' means floating decimal point---it can wash all
around.  With Floats, the exponent can range from about $-38 \rightarrow
+38$ and there is about 6 digits of precision.  You generate an array
using {\bf findgen} and a single variable by including a
decimal point ({\bf x=3.}) or using exponential notation ({\bf x=3e5}).

	Printing floating point numbers to the full native precision,
instead of what IDL regards as ``convenient'', 
requires using a \verb$format$ statement in the \verb$print$ (or, when
annotating a plot, the \verb$xyouts$) command. For example:

\begin{verbatim}
a= 1.23456789
print, a, format='(f20.10)'
\end{verbatim}

\noindent prints out 10 decimal points and 20 characters including the
decimal point.  Of course, we've defined \verb$a$ to higher precision
than single-precision float carries, so the last bunch of numbers beyond
the decimal won't be correct. To make that happen, you need\dots

\subsubsection{8 bytes: Double-Precision}

	Like Float, but the exponent can range from about $-307
\rightarrow +307$ and there is about 16 digits of precision.  You
generate an array using {\bf dindgen} and a single
variable by writing {\bf x=3d} or {\bf x=3d5}. Then, when you do the
following, it works:

\begin{verbatim}
a= 1.23456789d0
print, a, format='(f20.10)'
xyouts, xloc, yloc, string( a, format='(f20.10)')
\end{verbatim}

\noindent For annotating a plot with \verb$xyouts$, you explicitly
convert the number to a string of the specified format; the \verb$print$
statement does this too, but it's transparent because it automatically
does it for you.  

\subsection{COMPLEX NUMBERS}

A complex number $C$ consists of a pair of real numbers, one the real and
one the imaginary part [in IDL: {\tt real\_part(C)} and {\tt imaginary(C)}].
The two numbers are always floats and can be either single- or
double-precision. You generate a single-precision complex number $(2.2 +
i3.9)$ with: {\tt C = complex( 2.2, 3.9)} and a double-precision one with
{\tt dcomplex}. The arguments can be arrays, giving you a complex-number
array. There are also the complex analogues {\tt cindgen} and {\tt
  dcindgen} (like {\tt findgen} and {\tt dindgen}); and {\tt complexarr}
and {\tt dcomplexarr} (like {\tt fltarr} and {\tt dblarr}).


\subsection{ILLEGAL  NUMBERS}

As we've seen, there are limits to the size of digitally-stored
numbers. IDL follows the IEEE standard when calculations produce
out-of-bound numbers: \begin{enumerate}

\item A floating-point number that exceeds the upper limit is printed out
  as the string {\tt Inf} (but it is stored as a float).
  \item A float that is smaller than the lower limit is returned as zero
    and the statement ``Program caused arithmetic error: Floating
    underflow'' is written on the screen (unless the error reporting is
    suppressed).
\item An illegal number, such as $(-2)^{1/2}$, is printed out as the string
  {\tt NaN} (but it is stored as a float).
\item An integer number that exceeds either limit does not generate an
  illegal number; rather, it ``wraps around''. 

\end{enumerate}

Some IDL routines, such as the {\tt plot} routine, ignore illegal
numbers. Generally, however, you need a way to deal with these illegal
numbers---for example, so that you can exclude them from subsequent
mathematical operations. You do this with the {\tt finite} command. 

  \subsection {STRINGS} \label{strings}

	Strings store characters---letters, symbols, and numbers (but
numbers as {\it characters}---you can't calculate with strings! A string
constant such as {\bf hello} consists of five letters.  It takes 5 bytes
to store this constant---one byte for each character.  There are 256
possible characters for each of the bytes; with 2*26 letters (smalls and
caps) and 10 digits, this leaves 104 other possibilities, which are used
for things like semicolons and periods.  You can generate an array of
strings with {\bf strarr} and a single string with {\bf x = 'Hi
there!!!'}. 

Characters are represnted by byte numbers. You can easily find the byte
number for a particular character in IDL; for example, to find the byte
number that represents the semicolon character, type {\tt print, byte(';')}.
The answer: 59b. Similarly, to find the string variable represnted by the
byte number 59b, type {\tt print, string(59b)}. Quick---what's the answer?

\section{ ORGANIZATIONAL STRUCTURES}

\subsection{ SCALARS}

	A scalar is just a single number. For example, a string scalar
is {\bf joename= 'joe'}.

\subsection{ VECTORS}

	A vector is a one-dimensional array. For example, a
three-element vector of names is {\bf threenames = ['joe', 'ivan',
'mark']}.

\subsection{ ARRAYS}

	IDL handles arrays up to 8 dimensions, i.e. with 8 subscripts.
Arrays with two subscripts can be mathematically treated a matrices
using the $\#$ and $\#\#$ operators, and various matrix manipulation
routines; see IDL help under {\bf matrices} and {\bf matrix operators}. 
You create vectors and arrays using, for example, the {\bf fltarr} or
{\bf findgen} commands (for floating point numbers; equivalent commands
exists for all variable types). You populate them as appropriate, but
try to avoid using for loops; instead, use {\bf where}, appropriate use
of the {\bf *} operator, etc.

	IDL provides a great deal of flexibility in using subscripts to
address particular array elements, and this flexibility is what makes
IDL so useful. For example, consider a two-dimensional array {\bf
a=findgen(100,100)}. Then:

\begin{verbatim}
b = a[ 23:25, 67:69]
\end{verbatim}

\noindent makes {\bf b} a $3 \times 3$ 2-d array equal to a's 
array elements in the little box specified. The combination

\begin{verbatim}
indx = where( a gt 10.)
b = a[ indx]
\end{verbatim}

\noindent makes {\bf b} a 1-d array equal to the elements of {\bf a}
that are larger than 10. The combination

\begin{verbatim}
indx = where( a gt 10.)
jndx = where( a[ indx] le 100.)
b = a[ indx[ jndx]]
\end{verbatim}

\noindent shows that you can subscript arrays with other arrays, and
makes {\bf b} equal to a 1-d array equal to the elements of {\bf a} that
are both larger than 10 and less than or equal to 100.

\subsection{MATRICES}

        A matrix is just one step away from a 2-d array. IDL provides
all of the standard, and many sophisticated and advanced matrix,
operations. To multiply two matrices you use the $\#$ operator. Thus, the
matrix product {\bf C} of two matrices {\bf A} and {\bf B} is {\bf C} =
{\bf A} \# \# {\bf B}---or {\bf C} = {\bf A} \# {\bf B} depending on\dots a
tricky little point about matrices having to do with row-major or
column-major formats. 

        In a computer, a multidimensional data set can be indexed in two
ways, the {\it column-major} and {\it row-major} formats.  IDL uses the
row-major format, as does Fortran; the other major language, C, uses
column-major.  Suppose you have a $2 \times 2$ matrix called ${\bf A}$.
In IDL's row-major format, when you type [{\it print, A}] IDL prints

\begin{mathletters}
\begin{eqnarray}
\left[
\begin{array}{rrr}
A_{0,0} & A_{1,0} \\
A_{0,1} & A_{1,1} \\
\end{array}
\; \right] \; ,
\end{eqnarray}

\noindent which is different from (i.e., it's the {\it transpose of}) what
you are used to seeing in standard matrix notation which is the
column-major format

\begin{eqnarray}
\left[
\begin{array}{rrr}
A_{0,0} & A_{0,1} \\
A_{1,0} & A_{1,1} \\
\end{array}
\; \right] \; .
\end{eqnarray}
\end{mathletters}  

        We use the row-major convention such that when displayed in a
standard IDL {\it print} statement, they look correct. That means that
our definition is the transpose of the standard one.

        There are some matrix operations for which the difference is
important. This includes not only multiplication, but also some other
operations such as {\bf invert} and {\bf svsol}. IDL almost always
assumes that the inputs to these other operations follow our row-major
convention.

        If you want to be a purist and define the matrices in the
standard column-major manner, then go ahead and do so.  You then need to
do three things.  First, if you want to see the matrix displayed in the
usual way, then print its transpose by typing [{\it print,
transpose(A)}].  Second, in all our IDL matrix equations, replace ${\bf
\# \#}$ by ${\bf \#}$. Third, check any IDL procedure having a matrix as
input to see what it assumes (The default is almost always row-major).
        
        To be specific: if you follow our row-major convention, which is the
transpose of the standard one, then the matrix product must be written

\begin{mathletters}
\begin{equation}
{\bf C = A \ \# \# \ B}
\end{equation}

\noindent while, in contrast, if you follow the standard  
column-major one then you must write

\begin{equation}
{\bf C = A \ \# \ B}
\end{equation}
\end{mathletters}

          Why does IDL do this nonstandard thing? It's because it's more
straightforward for image processing, in which traditionally the images
are scanned row-by-row (as in a TV set) instead of column-by-column. And
IDL's origins are image processing, not matrix math. You might find
IDL's convention annoying when you're doing matrix math, but this is
more than compensated for by the intuitive feel you gain when doing
image processing.

\subsection{ STRUCTURES}

	Structures are immensely useful for any project in which data of
different types are related.  For example, if you have a catalog of
stars with positions and reddenings, you can put the whole catalog in a
structure array in which each element of the array contains many
quantities such as the name and position.  And you can have arrays of
structures.  Structures allow you to create and customize your own data
base. Having done this, using the {\bf where} command allows you
flexible access to anything with a one-line command.

We refer you to Chapter 7 of {\it Building IDL Applications} for
a complete discussion of structures.  Here we provide a quick
example. There are two types of structure, {\it anonymous} and {\it
  named}. In our experience, anonymous structures are usually preferred.

\subsubsection{Anonymous Structures}

Our abovementioned star catalog has several different data elements. We
group them all into a {\it structure}; each data element is called a
{\it tag}. Let's call the structure $A$. Then we define the structure
tags with the statement

\begin{verbatim}
A = {name: 'alpha ori', ra:5.3345, dec:-7.6568,reddening:fltarr(12)}
\end{verbatim}

\noindent Now if you type {\tt help,/struct,a} or {\tt help,/st,a}, you
will see on the screen

\begin{verbatim}
** Structure <752268>, 4 tags, length=72, data length=72, refs=1:
   NAME            STRING    'alpha ori'
   RA              FLOAT           5.33450
   DEC             FLOAT          -7.65680
   REDDENING       FLOAT     Array[12]
\end{verbatim}

\noindent The number \verb$<752268>$ is arbitrary and will change from
one anonymous structure to the next. The fact that a {\it number} appears
instead of a name means that $a$ is an {\it anonymous} structure. It
has four tags: the name, the two positions, and 12 different measurements
of reddening.  In IDL you refer to a particular tag, e.g.\ $dec$,
by typing {\tt a.dec}:

\begin{verbatim}
print, a.dec
\end{verbatim}

\noindent You could change things by typing

\begin{verbatim}
a.dec = 5.5
a.reddening[3] = 0.7
\end{verbatim}

\noindent Or you could could populate the 12
reddening measurements by typing, for example,

\begin{verbatim}
a.reddening = [1.2, 1.4, 1.3, 1.6, 1.3, 1.4, 1.3, 1.3, 1.6, 1.3, 1.4, 1.3,]
\end{verbatim}

You can redefine the tag names and number of tags in an
anonymous structure---but not for a named structure\dots

\subsubsection{Named Structures}

You can also create a {\bf named structure} by including the
name (here the name is ``star''):

\begin{verbatim}
b = {star, name: 'alpha ori', ra:5.3345, dec:-7.6568, reddening:fltarr(12)} 
\end{verbatim}

\noindent Now if you type {\bf help, b, /str} you see

\begin{verbatim}
** Structure STAR, 4 tags, length=72, data length=72:
   NAME            STRING    'alpha ori'
   RA              FLOAT           5.33450
   DEC             FLOAT          -7.65680
   REDDENING       FLOAT     Array[12]
\end{verbatim}

\noindent This says that the structure $B$ is a named structure with
name defined as {\it star}. We defined it to have the same tags as our
anonymous structure $A$ above. In contrast to an anonymous structure,
{\it Once a named structure is defined, you cannot change the names of
  the tags or the number of tags}.

\subsubsection{Arrays of Structures}

	If this were all, then structures wouldn't be
very useful because you have observed 585 stars, say, and you'd need a
separate structure for each. But you can create {\it arrays of structures},
e.g. for an anonymous structure we create the array {\it cataloga}

\begin{verbatim}
cataloga = replicate( a, 585) 
\end{verbatim}

\noindent and for our named structure we create the array {\it catalogb}

\begin{verbatim}
catalogb = replicate( {star}, 585) [OR catalogb = replicate( b, 585) ]
\end{verbatim}

\noindent These statements create structure arrays of 585 elements.  If
you type {\tt help, cataloga} IDL tells you this fact by printing on the
screen

\noindent \verb$CATALOGA        STRUCT    = -> <Anonymous> Array[585]$

\noindent and if you add the {\tt structure} keyword (typing {\tt help,
cataloga, /st}) you get, as above, 

\begin{verbatim}
** Structure <752268>, 4 tags, length=72, data length=72, refs=2:
   NAME            STRING    'alpha ori'
   RA              FLOAT           5.33450
   DEC             FLOAT          -7.65680
   REDDENING       FLOAT     Array[12]
\end{verbatim}

\noindent You refer to particular elements of the array in the usual
way, i.e.\ with an index number, but the index number comes directly
after the array name, like this:

\begin{verbatim}
print, catalog[ 3].ra
print, catalog[3].reddening[5]
cataloga[ 3] = a
\end{verbatim}

\noindent Or you can set values within a structure, for example

\begin{verbatim}
cataloga[ 3].name = 'carl heiles'
\end{verbatim}

	Try using structures when taking data for your experiments.
You'll grow to love them! 

\section{APPENDIX: HOW FLOATING-POINT NUMBERS ARE STORED IN COMPUTERS}

\subsection{The 32-bit IEEE standard}

Here we describe the 32-bit word case and assume the IEEE convention; see
{\it Numerical Recipes, \S 20.1}, for more complete information.  24 bits
are reserved for the mantissa and 8 for the exponent. Each reserves one bit
for the sign, leaving 23 and 7 bits for numbers. \begin{enumerate}
\item For the exponent, the the maxima and minimum numbers are $\pm 2^{7} =
  \pm 128$---or more accurately, --128 to +127.  This exponent applies to
  a binary number; converting to decimal, we have $2^{128} \approx 1.7
  \times 10^{38}$. Roughly, this is the maximum number that can be stored
  as a single-precision float.
\item For the mantissa, the maximum and minimum numbers are $\pm 2^{23}
  \approx 10^{7}$. This means that the fractional error in a stored number
  is about $10^{-7}$. As you do more and more calculations, using the
  results of one in the next, this fractional error increasees.
\item For double precision, the maximum number is about $10^{308}$ and
    the precision is about $10^{{-16}}$.
\end{enumerate}

\subsection{The routine {\tt machar}}

The IDL routine {\tt machar}, derived from the {\it Numerical Recipes}
routine of the same name, returns a structure that contains all the above
information. 



\enddocument 
\end



