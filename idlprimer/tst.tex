\documentstyle[11pt,psfig,aaspp4]{article} 
\begin{document} 
%\magnification=\magstephalf
%\parskip=0.2cm
%\parindent=1.0cm
%\def\simlt{\lower.5ex\hbox{$\; \buildrel < \over \sim \;$}}
%\def\simgt{\lower.5ex\hbox{$\; \buildrel > \over \sim \;$}}
%\tolerance=10000

\centerline{\bf IDL BASICS FOR ASTRO 250, SPRING 2002}
\centerline{\bf \today}

\noindent {\bf 1. BASIC INFO ABOUT IDL.}

	 IDL is extremely powerful and in wide use by scientists, and
especially astronomers.  Unfortunately, there are no good primers for
the beginner.  The best way to learn its basics is to go through our
tutorial idltut1 step-by-step; it takes you through the basic things,
including plotting.  Later, look at {\bf idldatatypes}, which tells you
about datatypes and organizational structures.  Finally, take a look at
{\bf idltut\_plot} (how to make ps files of plots); this uses some
routines in my personal library, see below) and {\bf plotting}.  For
more info, the book {\it Building IDL Applications} is indispensible. 

%\vskip{0.2in}
\noindent {\bf 2. THE IDL HANDIGUIDE}

	Make frequent use of the {\it IDL Handiguide} we distribute.  It
has the built-in IDL procedures arranged by topic.  So when you are
manipulating arrays, look under {\it Array Manipulation Routines} and
you may well find a procedure that does exactly what you want.  Ditto
for {\it Special Math Functions, Plotting Routines, Two-Dimensional},
etc., etc., etc.  

%\vskip{0.2in}
\noindent {\bf 3. THE GODDARD WEBSITE, ITS LINKS, AND ITS LIBRARY}

	One thing that makes IDL extremely useful for astronomers is
that lots of astronomers use it.  Long ago the Goddard folks wrote many
useful routines and more recently have assembled links to many other
libraries and reference material.  This is really a great resource: it's
huge, but things are categorized by subject to at least some extent. 

	In particular, you should print out the contents of {\it
http://idlastro.gsfc.nasa.gov/contents.html}.  It contains a list of the
Goddard library, categorized by topic.  For astronomers, the most
important section is {\bf Astronomical Utilities}.  Coordinate
conversions, time conversions, precession, Doppler shifts, etc., etc.,
etc.  Another is the {\bf FITS} sections, useful for all FITS files and
especially those from Space Telescope.  {\bf Image Manipulation}
contains many useful procedures for image analysis, such as FT-based
convolution, generating circular masks, maximum entropy, etc.  {\bf Math
and Statistics} has many procedures, some of which duplicate IDL's and
some of which do not.  {\bf TV Display Procedures} has nice procedures
that will draw boxes, circles, blink images, and more.  {\bf
Miscellaneous} has lots of procedures, of which I find the most useful
are {\it FINDPRO} (finds all occurances of a procedure in one's path!),
and {\it READCOL} (reads column-organized ascii data into IDL vectors). 

	Our system has the Goddard routines split according to
application category, located at
/usr/local/rsi/idl/ulib/external/idlastro/pro/ .  These versions are old
(1998 mostly) and ought to be updated---but sometimes Goddard routines
are not backwards compatible so you might want to follow my example and
keep your own version so that when you update you know what's happening. 
The {\it deep} group keeps a copy of an updated Goddard library at 
/deep2/dfink/cvs/idlutils/goddard .

\noindent {\bf 4. THE DEEP LIBRARY}

	Our (former) own Doug Finkbeiner and friends have developed an
extensive suite of clever, useful IDL procedures.  The include things
like the ability to write plotting captions in Greek letters using Latex
notation ({\it textoidl.pro}), a fancy image display routine that takes
positional info off of fits files and allows you to zoom, etc ({\it
atv.pro}) and a similar program for plotting one-d arrays such as
spectra ({\it splot.pro}), and programs for reading/writing fits files
that exceed the Goddard library program's capabilties.  You may want to
make use of them when you have learned the basics.  Some of these
routines are also referenced on Goddard's website. They are located at 
/deep2/dfink/cvs/idlutils .


\noindent {\bf 5. MY LIBRARY}

	Lots of people and many institutions have their own libraries. 
I have my own, and I will naturally use some of them in our class demos. 
It is located at /dzd2/heiles/idl/gen .  One feature I use all the time
is the easy ability to plot using color.  I suggest that, if you are
beginner, you use my IDL startup file.  You can do that by putting the
following line in your UNIX startup file (.cshrc)

\begin{verbatim}
setenv      IDL_STARTUP     ~heiles/idl/gen/idlstartup
\end{verbatim}

\noindent which will put my library, including my version of the Goddard
library, in your IDL path and define colors for plotting.  When you
enter IDL you will be prompted among four choices, which have to do with
the type of color display; until you know better, respond by typing {\it
pseudo}. 

\end{document}
\end


 
