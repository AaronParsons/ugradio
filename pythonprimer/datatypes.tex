\documentclass[psfig,preprint]{aastex}

%\documentstyle[11pt,psfig,aaspp4]{article}
\usepackage{listings}
\lstset{language=Python}

\begin{document}
\title{\bf QUICK PYTHON TUTORIAL NUMBER TWO: DATATYPES AND ORGANIZATIONAL STRUCTURES\\ \today}

\tableofcontents
	

By the term {\it datatype} we mean, for example, integers,
floating point variables, and complex numbers.  By the term {\it
organizational structures} we mean tuples, arrays and
dictionaries.  We cover these in the following sections.  All
available datatypes can be arranged in all available organizational
structures.  For example, we can have arrays of integers, dictionaries of
complex numbers etc\dots

\section {DATATYPES}
	Here we cover only the basic Python datatypes in brief here. 
There are others types like bytearrays, strings, unicode etc that we do 
not cover here.

\subsection{DIGITS, BITS, BYTES, AND WORDS}

	We have gotten to the place where you need to know a little
about the internal workings of computers. Specifically, how the computer
stores numbers and characters.

	Humans think of numbers expressed in powers-of-ten, or {\it
decimal numbers}.  This means that there are 10 digits $(0 \rightarrow
9)$ and you begin counting with these digits.  When you reach the
highest number expressible by a single digit, you use two digits and
generate the next series of numbers, $10 \rightarrow 99$.  Let $f$ and
$s$ be the {\it first} (1) and {\it second} (0) digits, respectively; then the
number is $f*10^1 + s*10^0$.  And so on with more digits. 

	Fundamentally, all computer information is stored in the form of
{\it binary numbers}, meaning powers-of-two.  How many digits? Two! They
are 0 and 1.  The highest number expressible by a single digit is 1. 
The two-digit numbers range from 10 to 11; the number is $f*2^1 + s*2^0$.  And
so on with more digits.  But wait a minute! The word ``digit'' is a
misnomer---it implies something about 10 fingers.  Here it's the word
{\bf bit} that counts.  Each binary ``digit'' is really a {\bf bit}.  So
the binary number 1001 is a 4-bit number.  What decimal number does the
binary number 1001 equal?

	For convenience, computers and their programmers group the bits
into groups of eight. Each group of 8 bits is called a {\bf byte}.
Consider, then, the binary number 11111111; it's the maximum-sized
number that can be stored in a byte. What is this number? 

	Finally, computers group the bytes into {\bf words}.  The oldest
computers dealt with 8-bit words---one byte.  Many present-day computers use 32-bit
words---four bytes.  Macs and the more capable computers deal with 64-bit
words---8 bytes. What's the largest number you can store in a 4-byte
word? And how about negative numbers?

	Below we describe how Python (and everybody else) gets around this
apparent upper limit on numbers.  They do this by defining different
data types.  Up to now, the details didn't matter much.  But now\dots We
don't cover all datatypes below---specifically, we omit a detailed 
discussion on the built-in datatypes in Python, but you can look these 
up if you are interested.

\section{NUMERIC DATATYPES IN PYTHON AND NUMPY}
\subsection{Native Numbers in Python}
	The datatypes {\it Python 2.7} natively supports are limited to 
long integer, float and complex. The datatypes are automatically fixed during 
variable assignment and need not be specified during declaration. However, 
remember that they do not change! The numeric datatypes are immutable, 
meaning that $1/2 = 0$ and not $0.5$. 

%Integer numbers that can be represented by up to 32 bits are 
%represented by the {\it int} datatype. They can be both signed and unsigned.
Integer numbers are represented by {\it long}. Unlike lower level 
languages (IDL, C), Python does not have a cap on the size of the integer. It 
is capable of allocating as many bits are required to represent the number. 
{\it Float} is used to represent fractions up to 64 bits. Check out the appendix to 
learn how fractions are stored.

	For most scientific computations you will proably use NumPy 
which has a lot more flexibility on the number of bits allocated to a number. 
This is more useful for programmers who deal with large sets of data (like radio 
astronomers!), and want to control the amount of memory used and optimise performance.

Note: The native datatypes in {\it Python 3} are a little different from 
{\it Python 2.7}. For example, division of integers is automatically 
cast to a float, which is accurate upto 15 decimal places.

\subsection {Integer Datatypes in NumPy}
	NumPy supports a much wider range of numeric datatypes than 
native Python. The length of the word and the precision of the fractional bit 
determine the memory required for representing a number. The shorter the word, 
the less memory required; the longer the word, the larger the numbers can be. 
Different requirements require different compromises.

\subsubsection {1 byte: The Boolean Datatype, \texttt{uint8} and \texttt{int8}}

	The \verb;bool; datatype is a single byte long and can be \verb;True; 
or \verb;False;. Empty string " " and $0$ are interpreted as \verb;False; and 
everything else is is considered \verb;True;. This datatype is most useful for
evaluating conditional statements.

	\verb;uint8; stands for unsigned integer that can be represented in 
8 bits.Therefore, its value runs from $0 \rightarrow 255$. \verb;int8; stands 
for signed integer. The first bit is reserved to store the sign of the number 
and the rest 7 bits are used to represent the integer. Therefore, its value 
runs from $-127 \rightarrow 127$.

You would think the byte datatype is not all that useful, but it is still used 
in computing! Computers store colors of pixels in bytes (1 bytes each for red, 
blue and green) and encode characters (like the string variable `a') in a byte.
Look up ASCII tables online if you want to learn more about character encoding.
Fun fact: Encryption algorithms often use a 1 byte key to scramble content 
before transmission on the internet. That is what makes https a ``secure" line.

%The byte data is not used as frequently anymore since computers have become faster. (cryptography)
%Python being a higher level language than Fortran, IDL or C stores both colors and 
%characters in custom formats. 
%These are still numbers under the hood, but the user 
%never sees these representations. The place where byte data is still frequently used 
%is in analog to digital convertors; most ADCs quantise the analog signal into 8 bits. 
%The one we have in the lab is more sophisticated and uses 16 bits.

If you ever need to generate a byte array, you can use

{\tt np.linspace(startvalue,stopvalue,dtype=np.uint8)}

Remember that if, during a calculation, a byte number exceeds 255, then it will {\bf wrap around}; for example, 256 wraps to 0, 257 to 1, etc.

\subsubsection{Multiple bytes}

        Most frequently for computation, you need numbers greater than $255$. NumPy 
offers integer representations with 2,4 and 8 bytes (16 bit, 32 bit and 64 bit numbers).
In each representation you can choose to have just positive numbers, called 
{\bf Unsigned Integers}. Unsigned numbers range from $0 \rightarrow 2^{B} -1$, where B 
is the number of bits. How do you think unsigned integers wrap around?

	Normally, you want the possibility of negative numbers and you can use 
{\bf Integers}. If B is the number of bits, the total number of integer values is $2^{B}$. 
One possible value is, of course, zero.  So the number of negative and positive values 
differ by one.  The choice is to favor negative numbers, so Integers cover the range 
$-2^{B/2} \rightarrow 2^{B/2}-1$. 

	In general, you can generate arrays of various sizes (8/16/32/64 bits) using 
{\tt dtype=np.uint<bits>} for unsigned integers and {\tt dtype=np.int<bits>} for 
signed integers. What are the limits on these numbers? See Python help 
under ``NumPy DataTypes'' for more information.  

\subsection {FLOATING DATATYPES IN Python} \label{floats}

	The problem with integer datatypes is that you can't represent
anything other than integral numbers---no fractions! Moreover, if you
divide two integer numbers and the result be fractional, but it
won't be; instead, it will be rounded down (e.g.  $5 \over 3$ is
calculated as 1).  To get around this, the {\it floating} datatype uses
some of the bits to store an {\it exponent}, which may be positive or
negative.  You throw away some of the precision of the integer
representation in favor of being able to represent a much wider range of
numbers. 

\subsubsection{4 bytes: Floats}

	``Floating point'' means floating decimal point---it can wash all
around.  You can have 16, 32 and 64 bit floats. With 32 bit floats (also 
called single precision float), the exponent can range from about 
$-38 \rightarrow+38$ and there is about 6 digits of precision.  You can 
generate an array by specifying {\tt dtype=np.float32} or by using 
exponential notation ({\tt x=3e5}).

	Printing floating point numbers to the full native precision,
instead of what Python regards as ``convenient'', requires using a 
\verb$format$ statement in the \verb$print$ command. For example:

\begin{verbatim}
a= 1.23456789
print("{0:.10f}".format(a))
print("{0:.20g}".format(a))
\end{verbatim}

\noindent prints out 10 decimal points or 20 characters respectively.  
Of course, we've defined \verb$a$ to higher precision
than single-precision float carries, so the last bunch of numbers beyond
the decimal won't be correct. To make that happen, you need\dots

\subsubsection{8 bytes: Double-Precision}

	The 64 bit float, allows the exponent to range from about $-307
\rightarrow +307$ and there is about 16 digits of precision.  You can 
generate an array using {\tt dtype=np.float64}. Then, when you do the
following, it works:

\begin{verbatim}
a= 1.23456789
print("{0:.10f}".format(a))
\end{verbatim}

\subsection{COMPLEX NUMBERS}

A complex number $C$ consists of a pair of real numbers, one the real and
one the imaginary part [in NumPy: {\tt np.real(C)} and {\tt np.imag(C)}].
The two numbers are always floats and can be either single- or
double-precision. You generate a single-precision complex number $(2.2 +
i3.9)$ with: {\tt C = np.complex64(2.2, 3.9)} and a double-precision 
one with {\tt np.complex128}. The arguments can be arrays, giving you a complex-number
array. There are also the complex analogues {\tt dtype= np.complex64} and {\tt
  dtype=np.complex128} that you can use in linspace to generate arrays.

Note: You can also create the native Python complex number using 
{\tt C = 2+3j}. Python has methods that you can call on complex numbers to access 
the real, imaginary parts, compute the conjugate etc. Try {\tt C.real}, 
{\tt C.imag} and {\tt C.conjugate()} on your computer.

For a more detailed discussion on numeric datatypes in NumPy see \url{https://docs.scipy.org/doc/numpy-1.13.0/user/basics.types.html}.

\section{ORGANIZATIONAL STRUCTURES}

Let us switch back from NumPy to Python and discuss some organization 
structures Python offers. These are only briefly described here to 
introduce you to the basics. You are encouraged to look up the Python 
documentation online which describes these in more detail and provides 
examples and tutorials.

\subsection{LISTS AND ARRAYS}
Lists in Python are a group of values enclosed in square brackets `[ ]'. They 
need not all be of the same datatype. Lists are mutable- you can add, 
remove values from the list dynamically. Python offers numerous methods on list 
objects that allows you to treat them like stacks, queues or multidimensional 
arrays and perform useful operations like sorting, filtering, searching etc.

Unfortunately, Python does not let you perform certain other useful 
operations like multiplying all the elements of a list with a scalar. For 
such purposes, you can either write a loop that iterates over all the 
elements in the list or you can use NumPy arrays. You can convert a native 
Python list to a NumPy array by using {\tt np.asarray(list)}. NumPy arrays are even 
more flexible and allow you to compute log, cosine, sin, mean and standard 
deviation of arrays. You can of course multiply the array by a scalar, but you 
can also multiply an array with another array! Is multiplying two arrays 
returning a matrix multiplication or an element-wise multiplication? NumPy arrays 
are about an order of magnitude more efficient than Python lists 
when it comes to array manipulation and computation.

If you ever forget whether the list of numbers you have is a Python list or 
a NumPy array, you can check by using {\tt type(list\_variable)}. Python lists 
show up as {\tt list} type and NumPy arrays as {\tt numpy.ndarray}. In fact, 
you can use the {\tt type} function on any variable in Python.

\subsection{TUPLES}
Tuples consist of a number of values, seperated by commas. They are a 
little like lists but not quite the same. Tuples are non-mutable, they 
cannot be changed like lists. You can identify tuples by the parentheses 
they are enclosed by `( )', unlike lists which are enclosed in square 
brackets `[ ]'. Even if you never use them, you will often encounter tuples 
as outputs of functions of the packages and modules you may use. For example, 
the color of a pixel in matplotlib is stored as a tuple of (r,g,b,a) where `a' 
quantifies the transperancy of the pixel.

\subsection{DICTIONARIES}
One of the most useful organizational structures Python offers is a dictionary. 
A dictionary is a key to value mapping. Instead of indexing by numbers 
(like lists), dictionaries index by keys. If you are familiar with C or IDL, 
these are best thought of as structures but more versatile. Dictionaries are 
enclosed by curly brackets `\{ \}' and the key:value pair are seperated by 
a colon. An example of a dictionary is given below:

\begin{lstlisting}
data = {}
data['metadata'] = {}
data['metadata']['signal_frequency_MHz'] = 13.25
data['metadata']['sampling_frequency_MHz'] = 62.5/2
data['metadata']['number_of_blocks'] = 5
data['metadata']['samples_per_block'] = 16000
data['metadata']['voltage'] = '1V'
data['samples'] = ugradio.pico.read_socket('1V')
\end{lstlisting}

You can then store the entire dictionary above to a file! This way you 
will not forget what the data you stored in a particular file is representing. 
Can you think of more innovative ways to use dictionaries? 

\section{APPENDIX: HOW FLOATING-POINT NUMBERS ARE STORED IN COMPUTERS}

\subsection{The 32-bit IEEE standard}

Here we describe the 32-bit word case and assume the IEEE convention; see
{\it Numerical Recipes, \S 20.1}, for more complete information.  24 bits
are reserved for the mantissa and 8 for the exponent. Each reserves one bit
for the sign, leaving 23 and 7 bits for numbers. \begin{enumerate}
\item For the exponent, the the maxima and minimum numbers are $\pm 2^{7} =
  \pm 128$---or more accurately, --128 to +127.  This exponent applies to
  a binary number; converting to decimal, we have $2^{128} \approx 1.7
  \times 10^{38}$. Roughly, this is the maximum number that can be stored
  as a single-precision float.
\item For the mantissa, the maximum and minimum numbers are $\pm 2^{23}
  \approx 10^{7}$. This means that the fractional error in a stored number
  is about $10^{-7}$. As you do more and more calculations, using the
  results of one in the next, this fractional error increasees.
\item For double precision, the maximum number is about $10^{308}$ and
    the precision is about $10^{{-16}}$.
\end{enumerate}


\enddocument 
\end



