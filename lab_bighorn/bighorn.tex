\documentclass[11pt,preprint]{aastex}
\usepackage{amsmath, amsfonts, amssymb}

\begin{document}

\title {21-cm LINE ASTRONOMY WITH A SSB MIXER}

This week we will measure the 21-m line (colloquially known among
astronomers as the {\it HI}---that's {\it H - ``Roman numeral I'',
pronounced ``H-one''}) line shape, velocity, and intensity using the big
horn on Campbell's roof.  Most of the measured power comes from our own
electronics, not the HI line.  It's often called ``noise''.  We need to
get rid of this instrumental contribution.  We need to get our {\it
intensities properly calibrated} and the {\it Doppler correction
properly applied}.  To this end, we want all groups to observe the same
position on the sky to see if they get the same answer (This has never
happened before in the history of this course!).  We want {\it each
group} to do the measurements and {\it each individual} to do the
reduction.

{\bf Plots are due April 12 in class!!!}

\section{The Measurements}

Begin by getting the data!  Before beginning measurements in earnest,
it's important to make sure the signal levels are OK and to
experimentally determine which way frequency increases on your power
spectra. So before getting out there in the rain to point the horn:
\begin{enumerate}

\item Set the system up as you will be using it to observe.  Take some
data.  How fast must you sample? Remember that our low-pass filter has a
$\sim 2$ MHz cutoff frequency.  What sample mode should you use for SSB
sampling? 

We recommend using the ``integer'' keyword for the sampler, which
returns integer numbers instead of voltages; this is faster. If you do
this you'll have to subtract the mean to get rid of the d.c.\ offset
before taking Fourier transforms. In fact, it doesn't hurt to always do
this, because there is usually an extraneous d.c.\ offset from
imperfections of various sorts. 

\item Look at the range of sample values by plotting a bunch of them. (If
  you have enough time, you could try using the \verb$histogram$ function
  for this).  The sampled numbers should cover plenty of bits; quantization
  should be only barely, or not at all, visibly evident in the histogram.

\item Insert a test signal so that it appears in the upper sideband and
take some data; then change the test signal frequency so it's in the
lower sideband.  You'll use this to determine whether the frequency axis
is flipped or not.

\end{enumerate}

Having determined that the system works, it's time to do astronomy!
Point the horn to Galactic coordinates $(l,b) = (120^\circ, 0^\circ$). 
The rotation matrix for converston between equatorial and Galactic
coordinates is in the coordinate-conversion handout.  We'll take three
separate measurements.  Take plenty of samples---the more you take, the
higher the signal-to-noise ratio.  Call this number NSAMP; you'll
calculate the power spectrum from these\footnote{To get really good
signal-to-noise, take lots (100?) of power spectra and average them}. 
\begin{enumerate}

\item Set the first local oscillator so that the HI line is roughly centered
  in the baseband spectrum. (The HI line frequency is 1420.4058 MHz).
  Because this is centered on the line, call this spectrum $s_{online}$.

\item Displace the first l.o.\ so that the line falls outside of the baseband
  spectrum. Because this is centered off the line, call this spectrum
  $s_{offline}$.

\item Repeat the offline spectrum with the cal on; call this one
  $s_{offline, calon}$.
\end{enumerate}

\section{The Analysis}

For the reduction: \begin{enumerate}

\item Make a plot or a histogram\footnote{To make a histogram, you can
  use IDL's native {\tt histogram.pro} or one of our wrappers:
  the simpler one {\tt histo\_wrap.pro} or the more general {\tt
  hist.pro} .} of some representative voltage values in one of your three
  datasets to make sure that they are, indeed, not too quantized and
  don't exceed the maximum range of the ADC. The histogram shape should
  look like a well-known function. Which function? Does it?

\item Make power spectra of the three measurements.  These spectra will
look EXTREMELY NOISY! And they will have a number of points equal to the
number of samples. Thus, the resolution (separation between adjacent
channels on the power spectrum) will be the total bandwidth divided by
NSAMP\footnote{By the way, what is the total bandwidth: is it the
sampling frequency or half the sampling frequency? Remember---your input
numbers are complex\dots}. This is much
narrower than needed. No HI line with a width of less than 6 kHz has
ever been detected, and for our purposes a resolution of 10 KHz is
plenty good enough. 

You can smooth over channels and reduce the noise. For example, suppose you
want to reduce the the NSAMP channels in the power spectrum $pspect$ to 512
channels. In IDL there are two
easy ways to do this: \begin{enumerate}

\item Use IDL's {\tt smooth} function: {\tt newspect = smooth(pspect,
    NSAMP/512)}. This degrades the resolution, but you will still have
  NSAMP channels in the output spectrum.

\item Use IDL's {\tt rebin} function: {\tt newspect = rebin(pspect, 512)}.
  This degrades the resolution and gives you 512 channels channels in the
  output spectrum.
\end{enumerate}

  In the smoothed $s_{online}$ spectrum you should be able to see the HI
line---together with the instrumental bandpass, and perhaps some other
bumps. The instrumental bandpass is determined mainly by the low-pass
filter, which should fall smoothly to zero as the frequency
increases. Does it? 

\item You can remove the instrumental bandpass, and thus get the shape
  of the line (but not the intensity), by taking the ratio 

\begin{equation}
ratio = {s_{online} \over s_{offline}} \ .
\end{equation}

\noindent This is the {\it first} factor (i.e., the shape) in equation
(13) in the spectral-line handout.

\item To get the line intensity in terms of the calibration noise
  source, multiply the ratio spectrum by $T_{sys}$---the second factor
  in equation (13). We obtain $T_{sys}$ by combining the calon and
  caloff spectra, as per equation (15) in the handout:

\begin{equation}
T_{sys} = {\sum s_{offline} \over \sum(s_{offline, calon} - s_{offline})} T_{cal}
\end{equation}

\noindent where the numerator and denominator are summed over all of the
channels, as in equation (15). We think $T_{cal}$ is about 20 K.  Then
the final, intensity-calibrated spectrum is equation (16) in the
handout, namely
\begin{equation}
final \ calibrated \ spectrum = ratio \times T_{sys}
\end{equation}

\item Plot your final spectrum versus the r.f.\ frequency.

\item Plot your final spectrum versus the Doppler
  velocity\footnote{Astronomers usually express velocities in km s$^{-1}$.}.
Remember that, by astronomical convention, positive velocity means
motion away (remember the expansion of the Universe!), so

\begin{equation}
{v \over c} = -{\Delta f \over f_0} \ ,
\end{equation}

\noindent where $c$ is the speed of light and $\Delta f$ is the frequency
offset from the line frequency $f_0$\footnote{Radio astronomers, being
  frequency-oriented, use this equation; optical astronomers, of course,
  use something different: ${v \over c} = {\Delta \lambda \over
    \lambda_0}$.}. 

\item Calculate the Doppler correction using the {\tt ugdoppler.pro}
(for documentation use {\tt doc} or {\tt doc\_library}). Correct the
velocities to the Local Standard of Rest (LSR) and make a third plot.

{\it Notes on \verb$ugdoppler$}: \begin{enumerate}

\item You need the Julian day of the observation, which you get from
  IDL's {\tt JULDAY} function. The Julian day is defined for {\it
  Greenwich}, whose time is 8 hours ahead of PST (and 7 hours ahead of
  PDT!). That is, if the time in Berkeley is 6 pm (that's 18 hours) on
  March 17, then the time in Greenwich is 1 hour on March 18, and that's
  what you use as input to {\tt JULDAY}. To obtain the Julian day right
  now, use {\tt systime( /julian, /ut)}. 

\item You need the observatory coordinates (north latitude and west
  longitude) in degrees; you could enter them with the pair of optional
  input parameters {\tt (nlat, wlong)}, but you don't have to because the
  default values are Campbell Hall's values.

\end{enumerate}
\end{enumerate}

Hand in the plots together with relevant commentary.


\end{document} 

