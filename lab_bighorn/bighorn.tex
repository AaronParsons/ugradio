\documentclass[11pt,preprint]{aastex}
\usepackage{amsmath}
\usepackage[top=1in, bottom=1in, left=1in, right=1in]{geometry}
%\usepackage{natbib}
%\usepackage{natbibspacing}
\usepackage{enumitem}
\usepackage{url}
\setlist[itemize]{noitemsep, topsep=0pt}
\setlist[enumerate]{noitemsep, topsep=0pt}
%\setlength{\bibspacing}{0pt}
\setlength{\parskip}{0pt}
\setlength{\parsep}{0pt}
\setlength{\headsep}{6pt}
\setlength{\topskip}{0pt}
\setlength{\topmargin}{0pt}
\setlength{\topsep}{6pt}
\setlength{\partopsep}{0pt}
\setlength{\footnotesep}{8pt}

\begin{document}
\tolerance=10000
\def\simlt{\lower.5ex\hbox{$\; \buildrel < \over \sim \;$}}
\def\simgt{\lower.5ex\hbox{$\; \buildrel > \over \sim \;$}}

\title {LAB 2: Astronomy with the 21-cm Line and Waveguides}

\tableofcontents

\section{Introduction}

\noindent
In this lab, we build upon the signal processing techniques of the previous lab,
applying them to making astronomical measurements with a horn antenna
tuned to frequencies around 1.4 GHz where we can measure 21-cm hyperfine
emission from neutral hydrogen in the Milky Way.  With luck, we should be able
to see evidence of the spiral-arm structure of gas in our galaxy.

In the second half of this lab, we expand our understanding of how signals
propagate with measurements from transmission lines and waveguides. Our first examination
of statistical regression uses these measurements to fit a line that empirically
determines the speed of light.

As before, your work should be oriented around producing a high-quality report that
addresses the goals in \S\ref{sec:goals}. From your previous report, you will
be familiar with the need to succintly describe your experimental setup and hardware.
New to this report will be sections describing how you obtained your observations
and the calibration and analysis that went into producing your key results.

We have a few more suggestions for how to organize your work, building on the 
suggestions of the previous lab:
\begin{itemize}

\item Develop (version-controlled) scripts for taking data from the command line.
Have these scripts save data with {\bf all the necessary metadata} needed to 
reconstruct your observation. Omitting critical information could
mean you have to redo observations later.

\item Test your reality. How do you know your rotation matrices give you the
right pointing? How do you know if your gain calibration is right? To build
your confidence, you need internal consistency checks.

\item As you conduct observations, think critically about your assumptions and
approximations.  How well have you pointed the horn? How completely did
you cover the aperture for the calibration run? The more you can capture
as you do the work, the more resources you will have for critically analyzing your
results and procedures in your report.

\item By the end, you should have empirical numerical results to report. Remember
that values without errors are meaningless. As you estimate errors, think about how 
well your model matches the data. How can you estimate the
accuracy of your results? Are you confident enough in your analysis to
bet money on your reported accuracy? If not, think about the sources of error
(systematic and otherwise) that may be undermining your confidence.

\end{itemize}


\section{Goals and Instructions for Your Report} \label{sec:goals}

\noindent
As always, lab reports and analysis code must be written {\it individually}. 

The electronic components are such an important part of this
lab that you should include a block diagram of the telescope electronics
in your lab report. You can prepare this either by either hand or
computer. If by hand, you'll have to scan your drawing to get a file you can
insert into your report's tex file. If by computer, you can use your
favorite software for making line drawings.  Although it is dated, {\tt xfig}
comes installed on Linux, so that's an option.

Below is a list of learning goals that your report should demonstrate mastery of.

\begin{itemize}

\item Learn about time. Demonstrate accurate conversion between UTC, PST, LST,
and Julian Day.

\item Learn about telescope pointing and use rotation matrices to convert among spherical
  coordinate systems.

\item Measure a 21-cm line power spectrum from atomic hydrogen in the
  Milky Way at a defined and reproducible location.

\item Calibrate your telescope observations to an absolute scale, remove
systematic instrumental effects in your spectra, and apply noise-reduction
techniques.

\item Learn about Doppler correction and produce spectra with velocities 
calibated relative to standard frames of reference.

\item Fit spectra with Gaussian components to localize celestial structures.

\item Demonstrate undestanding of transmission lines and waveguides, and 
how to minimize reflections with impedance matching.

\item Derive propagation velocities of radio waves by measuring the wavelength of
  standing waves.

\item Empirically explore the relationship between waveguide geometry and cutoff frequency.

\item Use linear and non-linear least-squares methods to fit models to data and
estimate error.

\end{itemize}

\section{Schedule}

\noindent
This lab covers a lot of new territory, particularly in the realm of data
analysis. Work hard to get good observations early, reserving plenty of
time for analysis and, if necessary, re-observation. A suggested schedule
follows.
\begin{enumerate}

\item {\it Week 1.}
Finish \S\ref{radioastro} and \S \ref{analysis}, and understand \S
  \ref{jultime}. Be prepared to show work, software, and results in class.

\item {\it Week 2.}
Finish \S \ref{meas2}, \S \ref{expt}, and \S \ref{secondweek}. Produce
candidate plots and analysis results for class.

\item {\it Week 3.} Finish any re-observations needed for the first
two weeks, then write your formal
  report. Remember to follow the report guidelines and address the specific
goals in \S\ref{sec:goals}.

\end{enumerate}


\section{What Time Is It?} \label{jultime}

\noindent
From relativity, we know that time depends on
reference frame. Nothing drives this home like sitting on the surface
of an orbiting, spinning sphere.
To deal with this, humans have invented a surfeit of time standards.
We will restrict our attention to those most useful for astronomy.

{\bf Coordinated Universal Time (UTC)} is the Civil Time\footnote{You use
  Civil Time when you set your alarm clock for getting up in the
  morning.} in Greenwich, England, which is 8 hours ahead of 
Pacific Standard Time (PST). In the middle of this course, we switch from
PST to Pacific Daylight Time (PDT); 
${\rm PST}={\rm UTC} - 8$ hr, while ${\rm PDT}={\rm UTC} - 7$ hr). UTC measures
solar time; 24 hours is the time it takes for the Sun to appear
in the same position in the sky on neighboring days.

Another standard, common since the computer era, is {\bf Unix Standard Time}, which measures the
number of seconds since midnight 1 Jan., 1970, UTC. 
This clock is how your computer keeps track of time. All other computer times
are derived from this one; it
is what you get if you call {\tt time.time()}.
You phone/computer uses a quartz crystal oscillator to keep track of 
this time over short intervals.  Over longer intervals, a time-exchange protocol called 
Network Time Protocol (NTP) is used to discipline the on-board clock to 
atomic clocks in timing centers around the world.

{\bf Sidereal Time} is ``star time''; it tells when
distant stars (i.e. not the Sun) rise and set. The sidereal time period is
shorter than the 24-hour Solar time period: $1/356.24$ less, to
be nearly exact. The difference, which corresponds to slightly 
less than 4 minutes per day, comes from how much the Earth's orbit around the
Sun changes the position of the Sun in the sky.
Sidereal time depends on the Earth's spin. The
Earth is constantly slowing owing to tidal friction produced by the
Moon. (Where does the angular momentum go?) The conversion between Civil Time and LST has to be adjusted
periodically by inserting {\it leap seconds}.

{\bf Local Sidereal Time} (LST)
adjusts sidereal time by longitude so that a star of a given {\bf right ascension} 
(RA; the ``longitude" coordinate for celestial sources) transits the local meridian
(the line of longitude that goes right overhead) at the moment
that ${\rm LST}={\rm RA}$. 

The {\bf Julian Day} is a sequential numbering of Solar days since 1 Jan. --4713. 
It
uniquely specifies the date and time (as a fraction of a day) without 
involving 
months (with their nonsensical definitions\footnote{``Thirty days
  hath September\dots''}), leap years, etc. It begins at noon 
in Greenwich. This makes it 12 hours out of phase with UTC, but the
international date line is 12 hours away from UTC, so JD is in sync
with humankind's definition of `when the day begins'. 
On computers, the Julian day is represented as double-precision
float.

The {\bf Modified Julian Day} is a shorter version of Julian Day. MJD is (sigh) 12
hours out of sync with JD, so it is in sync with UTC.
MJD=0 corresponds to 0 hr UT on 17 Nov 1858.

\subsection{ Useful Python Procedures for Time Conversion}

\noindent
Python has several built-in modules for dealing with time, including {\tt time} 
and {\tt datetime}.
Astronomers, however, have historically been the best time-keepers on the planet.  For
astronomy-quality time routines, {\tt astropy.time}, part of the {\tt astropy} package, is
the industry standard.  We have wrapped some of them into {\tt ugradio.timing} for
your convenience.

Calculating LST requires knowing your
longitude. New Campbell Hall (NCH) is at 
{\tt (lat,lon) = (37.8732, -122.2573)},  Note that {\tt lon = -122.2573 = (360 - 122.2573) = 237.7427}. 
This information is stored in the {\tt ugradio.nch} module.
For all procedures mentioned below, this location is default

\begin{verbatim}
local_now = ugradio.timing.local_time() # current local time as a string
ut_now = ugradio.timing.utc() # current UTC as a string
ut_now = ugradio.timing.unix_time() # seconds since 1 January 1970
julian_now = ugradio.timing.julian_date() # current julian day (which \
        contains the current time, too--it's not just an integer \
        number. 
lst_now = ugradio.timing.lst() # current LST at NCH
lst_julian = ugradio.timing.lst(jd) # LST for the specified Julian day                       
ut_julian = ugradio.timing.unix_time(jd) # seconds since 1/1/1970 for given JD
julian_ut = ugradio.timing.julian_date(ut) # julian day for given unix time
\end{verbatim}

\section{Handouts} \label{handouts}

\noindent
The following handouts may be of use to you:
\begin{enumerate}
%\item A second-level IDL tutorial. {\tt idldatatypes.pdf} ``QUICK IDL % XXX update for python
%  TUTORIAL NUMBER TWO: IDL DATATYPES AND ORGANIZATIONAL STRUCTURES'' For
%  this lab you use complex numbers, matrices and matrix
%  math, and structures. This handout describes the complete set and, in
%  addition, how numbers are stored and represented, and how accurate
%  they are. {\it You need to know all of this stuff!}

\item The complete story of producing calibrated spectral line shapes
  and intensities: {\it CALIBRATING THE INTENSITY AND SHAPE OF SPECTRAL
  LINES}. This handout is optional because the steps discussed in
  the lab instructions are sufficient, but
  if they are unclear or you want to know more, this writeup has it all.

\item How does the power we receive with a telescope relate to a source being
  observed? It's all about specific intensity, usually denoted $I$:
  ``SPECIFIC INTENSITY: THE FOUNT OF ALL KNOWLEDGE!''. 
  You need to understand these topics to interpret your measurements.

\item Astronomical coordinate transformations with Rotation Matrices:
  ``SPHERICAL/ASTRONOMICAL COORDINATE
  TRANSFORMATION''. converting among Az/El, Ha/Dec, Ra/Dec, L/B. 
    You need to know all of this stuff!

\item You can do this lab without doing any textbook-type reading\dots
  but then you wouldn't gain the insight from the highly readable and
  well-written text by Ramo, Whinnery, and van Duzer (RWvD), {\it Fields
    and Waves in Communication Electronics}. This is a great book
  because it contains commentary on applications to modern devices
  (e.g., the discussions of transmission lines carrying pulses from one
  piece of computer hardware to another are very nice).  Reading this
  book is optional, but in any case see notes below in \S
  \ref{RWvD}.

\end{enumerate}

\section{Your First 21-cm Measurement (Group Lab Activity, Week 1)}
\label{radioastro}

\noindent
This week we will use the techniques we explored in Lab 1 to observe the
21-cm line of neutral hydrogen (colloquially, {\bf
  HI}, pronounced ``H-one''), measuring its line
shape, velocity, and intensity using the big horn on the New Campbell Hall roof.
Most of the measured power comes from noise in our own electronics, not the HI
line.  It's often called noise, and we need to calibrate out this
instrumental contribution and the responses of our amplifiers and filters to
obtain the correct line shape and intensity.

We'll measure the 21-cm line twice. The first time, the goal is to master
the technical aspects and familiarize ourselves with the system and
procedures, so instead of worrying about where to point the horn, we'll
just take whatever position happens to be overhead. The second time (\S
\ref{meas2}), we'll manually point the horn to a designated position and
make a calibrated profile to compare with a well-established standard
profile measurement.

\subsection{The Receiving System}

\noindent
There are two options for receiving a signal from our horn antenna. The
first option, which is probably the easiest, is to use the 
Raspberry Pi / SDR system with its built-in digital downconverter. The
second option is to build the downconverter yourself out of analog
components, and then use the PicoSampler to acquire the I/Q data. We
will describe both below:

\subsubsection{Digital Down-Conversion}

\noindent
Having mastered the intricacies of DSB/SSB mixers and filters in the
first lab, we are now in a position to jump to a higher level of 
abstraction with dedicated hardware for down conversion.

The NeSDR SMART modules we have in the lab are based on two chips:
the R820T2, which contains a voltage-controlled oscillator and mixer
(i.e. a DSB with a tuneable LO), and the RTL2832U, which contains
a 28.8 Msps ADC, two multipliers (mixers) fed with digital sine/cosine
waves, and on-board FIR filters (convolution filters). 
In other words, the RTL2832U contains an all-digital SSB mixer/filter
down-converter.

Using the {\tt rtlsdr} driver that has already been installed on your
RPi and the {\tt pyrtlsdr} bindings for Python control, you can set
the LO frequency of the R820T2 DSB mixer and the RTL2832U SSB mixer.
You can also (as you may have discovered in the first lab) adjust the
sample rate at the output of the RTL2832U, which controls how many of
the 28.8 Msps samples get through. This is called down-sampling, and
it is necessary to get the data bandwidth down far enough that a
USB interface can carry it over to the RPi.

For this lab, we have used the {\tt pyrtlsdr} module to create a
function for you, {\tt ugradio.sdr.capture\_data\_mixer}, which can be
used to control both the LO frequency and the output sample rate of
the NeSDR module. You are welcome to use this function to acquire data,
or you can read the code it uses underneath (or the documentation of
the {\tt pyrtlsdr} module) to write something that runs faster to
get {\it all} of the data coming across the USB interface to build
sensitivity faster.

We suggest setting your LO off to the side of the 1420.405 MHz line,
since the 0-frequency bin in your spectrum almost always has an
artificial spike in it. How far you can tune your LO away from the
rest frequency depends on the sample rate (read: Nyquist cutoff) you
choose.

\subsubsection{Analog Down-Conversion}

\noindent
We can alternately build a a double-heterodyne system 
for the 21-cm line in analog.
{\bf Double heterodyne} means we have two mixing stages.

For the first mixer, we use the DSB technique. Suppose we
  use an LO (the first local oscillator) at 1230.0 MHz so that the
  difference frequency is 190.4 MHz and the sum frequency is 2650.4 MHz.
  We use a 20 MHz-wide bandpass filter centered at 190 MHz to excise
  the sum frequency, leaving us with one copy of the 21-cm line
  shifted to 190.4 MHz from 1420.4 MHz.

For the second mixer, we use the SSB technique. The second
  LO has frequency 190.0 MHz, so the output is at {\bf baseband},
  with positive and negative frequencies centered about zero. In the
  absence of Doppler shift, we are left with the line centered at 0.4
  MHz. However, the line is shifted and broadened by Galactic rotation,
  the Earth's orbital velocity, and the Earth's spin, so it covers a range of frequency
  of, for example, $1420.2 \pm 0.5$ MHz. Use a low-pass filter with a
  cutoff frequency of $\sim$2 MHz to
  eliminate aliasing, sample the complex signal, and Fourier transform
  to calculate the power spectrum. We take many power spectra and
  average them to reduce the noise.


\subsection{Your Measurement}

\noindent
Before taking data, do basic system
checks: (1) make sure the signal levels are okay and (2) experimentally
determine which way frequency increases on your power spectra. In order
to test that you are seeing data through an active (and powered!) signal
chain, you can use the tone-injection system to broadcast a known
signal at RF frequencies that you can receive and cross-check the
frequency of. Make sure you are seeing the actual signal by moving
it around in frequency, turning it on/off, or changing its amplitude
to make sure the ``birdie" you put in is where you expect it in
your spectrum.

Once you are convinced your signal is getting through,
set the system up as you will be using it to observe. Point the
 horn to zenith to reduce interference and thermal noise.  Take some
data.  How fast must you sample? 

Look at the range of sample values using, e.g.,
  {\tt numpy.histogram}.  As in Lab 1, check that you are not overly quantized or clipping.
  Adjust {\tt volt\_range} in the call to {\tt pico.capture\_data}
  accordingly (or use analog attenuators and/or 
digital gain in {\tt sdr.capture\_data\_mixer}).
  The histogram shape should look like a well-known function. Which
  function? Does it?

Again, insert a test signal so that it appears in the upper sideband and
take some data; then change the test signal frequency so it's in the
lower sideband.  You'll use this to make sure that the SSB mixer is
doing its job properly and to determine whether the frequency axis
is flipped or not.

As a final note, remember that if it rains, the horn is a gigantic bucket. 
You won't see any astronomical signal unless you dump it!

\subsubsection{Planning Observations}

\noindent
Having determined that the system basics work, it's time to do
astronomy! For this first measurement the main goal is to master the
technical aspects, so use whatever position happens to be overhead and
point the horn straight up. The line will be strongest---strong enough
to see visually---in the range LST = 19 -- 6 hr.

It is most convenient to use temperature units for the power
that we measure. Accordingly, the power that we measure is called the
{\bf system temperature} $T_{\rm sys}$. It's a function of frequency and has two
kinds of behavior: the {\bf continuum}, which is devoid of spectral features
and changes very slowly with frequency; and the {\bf line}, which in this
case is the 21-cm line and it changes relatively rapidly with frequency---hence our
desire to obtain the line shape.

The system temperature has two contributions: the dominant contribution
  from our electronics, which we call the {\bf receiver temperature} $T_{\rm rx}$;
  and the contribution our antenna picks up from the sky, the {\bf sky temperature}
  $T_{\rm sky}$.  (This is also sometimes called the {\bf antenna temperature}).  
 Thus $T_{\rm sys}= T_{\rm rx} + T_{\rm sky}$, and above 300 MHz, usually $T_{\rm rx} \gg T_{\rm sky}$. 

Our horn is equipped with an old, noisy first amplifier so $T_{\rm rx} \sim
300$ K; in contrast, our Leuschner telescope is much better, with $T_{\rm rx}
\sim 50$ K. The sky temperature comes from the Cosmic Microwave Background,
with brightness temperature ($T_{\rm CMB} = 2.7$ K); from
interstellar/intergalactic space, with brightness temperature
$T_{\rm IGM}$ (usually no more than a few K in the continuum and up to
about 100 K in the HI line); and the Earth's atmosphere with brightness
temperature $T_{\rm atm}$, perhaps a few K at the HI line frequency. So
off of the HI line we have $T_{\rm sky} \sim 10$ K, while on the HI line, in the
Galactic plane where it is strongest, we have $T_{\rm sky} \sim 100$ K.

We'll take two sets of data:
\begin{enumerate}
\item a long integration to measure the line's
{\it shape}, and 
\item a short integration so we can calibrate the line's
{\it intensity.}
\end{enumerate}

\subsubsection{Two Frequency-Switched Line Measurements}

\noindent
The measured power spectrum shape is dominated by the
frequency-limiting filters acting on the system temperature. To see the
line, which is weak, we need to correct for these filter shapes, which
we do by obtaining a spectrum containing no line. Accordingly, to get
the shape we take two spectra: one with the line present (the on-line
spectrum $s_{\rm on}$) and one with the line not present (the off-line
spectrum $s_{\rm off}$).

We could obtain the on-line
  spectrum by centering the line and making a measurement; then changing
  the first LO frequency so that the line shifts either completely
  outside the band (this is called {\bf frequency switching}), or partway
  over but is still in the band (in-band frequency
  switching). The latter is better because you are always looking at
  the line; you end up with more measurements and better signal/noise.

To accomplish this, take a spectrum with the line roughly in the upper
half of the baseband spectrum, and another with it centered roughly in
the lower half. Use the first as the on-line and the second as the
off-line spectrum for the upper half. Similarly, for the lower half,
use the second as the on-line and the first for the off-line.  (The HI
line frequency is 1420.4058 MHz).

%You can change the LO frequency in software: use {\tt ugradio.agilent.SynthClient}. 

The line is weak, and you'll need to take lots of spectra. Recall that
{\tt pico.capture\_data} obtains a specified number of blocks ({\tt nblocks}),
each of which has 16000 datapoints and writes them all out into a single
file. Suppose you decide to obtain 10000 spectra. You might be tempted
to do it in one fell swoop by setting {\tt nblocks=10000}. That
may work, but be aware that it will take about 5 minutes or more. You
might want to instead take ten sets with {\tt nblocks=1000} so you can do a
sanity check on your data without waiting forever. Similarly, if
you are using {\tt sdr.capture\_data\_mixer}, you'll want to grab many
chunks of data. If you write well-optimized code (which may involve
using the streaming capability that {\tt pyrtlsdr} provides), it is
possible to retrieve all of the data from the NeSDR without missing a
sample; this will enable to you build sensitivity as fast as theoretically
possible for the analog system you are using.

\subsubsection{Two Frequency-Switched Reference Measurements}

\noindent
Intensity calibration requires a second set of (short) measurements
  with your telescope aimed toward a source of known temperature.
  Easiest is probably to take one with the horn
  looking at a known blackbody and one looking at the cold sky. What's a
  convenient blackbody? You and your friends! So take one short
  measurement with the horn pointing straight up at the cold sky and the
  other with as many people as you can find standing in front of it to
  fill the aperture. Call these spectra
  $s_{\rm cold}$ and $s_{\rm cal}$, respectively.


\section{Analysis (Individually at Home, Week 1)} \label{analysis}

\subsection{Take a Suitable Average/Median}

\noindent
Consider, first, the $s_{\rm on}$ and $s_{\rm off}$ spectra, from which
you can find the line shape. There are many individual spectra, 10000 in
our above example. You need to combine these to make a single spectrum
for each measurement. You can do this by averaging the power spectra
(use {\tt numpy}'s {\tt mean} function) or by taking the medians (use {\tt numpy}'s
{\tt median} function). The former gives a less noisy result, but the
latter handles time-variable interference better; use both and compare
the results.

Even after combining the 10000 spectra, the resulting
spectrum will look noisy.  You can reduce the noise by averaging
over channels---by `smoothing' the spectrum. This reduces the noise, but
degrades the spectral resolution, so you have to make a compromise on
how many channels to smooth over. To decide, realize that the HI line is
never narrower than about 1 km/s, so it's OK to degrade the frequency
resolution to, say, 1 or 2 kHz. Again, you do the smoothing by averaging
(by using {\tt numpy.mean} ) or medianing (by using {\tt
  numpy.median}), and again try both to see what happens.

In the smoothed $s_{\rm on}$ spectrum you might not be able to see the
HI line, because the instrumental bandpass dominates the spectrum
shape. The instrumental bandpass is determined mainly by the low-pass
filter, which should fall smoothly to zero as the frequency
increases. Does it? If not, should you worry about aliasing?

\subsection{Get the Line Shape}

You can remove the instrumental bandpass to get the shape
of the line $s_{\rm line}$ (but not the intensity), by taking the ratio 

\begin{equation}
s_{\rm line} = \frac{s_{\rm on}}{s_{\rm off}} \ .
\end{equation}
%
This is the first factor (i.e., the shape) in equation
(13) in the spectral-line handout.
%
\subsection{Get the Line Intensity}

\noindent
To get the line intensity in temperature units, we need to multiply terms of the calibration noise source,
multiply the shape spectrum by the {\bf gain}---the second factor in
the handout's equation (13). We obtain the gain, $G$, by using a known
difference in system temperature between the 
the calibration ($T_{\rm sys,cal}$) and the cold sky ($T_{\rm sys,cold})$
measurements, and seeing how much that known temperature difference changes
the measured values in your spectra,  as per equation (15)
in the handout:
%
\begin{equation}
G =
\frac{T_{\rm sys,cal} - T_{\rm sys,cold}}{\sum(s_{\rm cal} - s_{\rm cold})} 
\sum s_{\rm cold} 
\end{equation}
\noindent Here, a sum is over all channels in a spectrum. All this
equation does is to convert measured units, which are digital numbers
from the system, into physically meaningful units, i.e.\ Kelvin. Here,
$T_{\rm sys,cal} = 300$ K, because that's the thermal power you injected by
standing in front of the horn. Since $T_{\rm sys,cold} \ll
T_{\rm sys,cal}$, to a first approximation you can neglect it (which is what we
did in the handout). Then the final, intensity-calibrated spectrum is
equation (16) in the handout, namely
\begin{equation}
T_{\rm line} = s_{\rm line} \times G
\end{equation}
%
\subsection{Plotting Intensity vs.\ Frequency---and Velocity}

\noindent
First,
plot your final calibrated spectrum versus the r.f.\ frequency. Next,
plot it versus the Doppler velocity\footnote{Astronomers usually express
  velocities in km s$^{-1}$.}.  Remember that, by astronomical
convention, positive velocity means motion away (remember the expansion
of the Universe!), so

\begin{equation}
\frac{v}{c} \approx -\frac{\Delta \nu}{\nu_0},
\end{equation}

\noindent where $c$ is the speed of light and $\Delta \nu$ is the frequency
offset from the line frequency $\nu_0$\footnote{Radio astronomers, being
  frequency-oriented, use this equation; optical astronomers, of course,
  use something different: $\frac{v}{c} = \frac{\Delta \lambda}{\lambda_0}$
   At high redshifts the difference becomes significant;
  the optical definition is the usual standard.}. 

\subsection{Finally, Choose a Reference Frame}

\noindent
You may think we've done it all by this point, but we haven't! We need
to correct the observed velocity for the orbital velocity of the Earth,
and also the Earth's spin. And when observing the Galaxy, it is
customary to express velocities with respect to the {\bf Local Standard of
Rest (LSR)}, so that's yet another correction.

Calculate the Doppler correction using {\tt ugradio.doppler.get\_projected\_velocity} 
Correct the velocities and compare the spectrum for the observing frame
and the LSR frame, which is approximately the frame that would rotate
around the Galaxy in a circular orbit.  Correcting to the LSR involves many components,
including primarily: the rotation of the observatory around the center of the Earth, the
orbit of the Earth around the barycenter of the solar system, and the peculiar velocity
of our Sun with respect to other stars in the neighborhood.  There are higher-order
corrections (which are in the {\tt barycorrpy} package we are using in {\tt ugradio.doppler},
see (Wright \& Eastman, 2014), including 1) special relativistic treatment of velocity,
2) general relativistic effects from the influence of the gravitational fields of all bodies
in the solar system, and 3) the proper motion of the target source.

{\it Notes on \verb$ugradio.doppler$}: \begin{enumerate}

\item You need the celestial coordinates of the source, $(ra, dec)$. How
  to find these for the horn pointing straight up? You could use
  rotation matrices. However, when looking straight up you don't need
  this powerful technique because, quite simply, the Dec is equal to the
  Latitude and RA=LST.

\item You need the Julian day of the observation; see \S \ref{jultime}.

\item You need the observatory coordinates (north latitude and west
  longitude) in degrees; you could enter them with the pair of optional
  input parameters {\tt (obs\_lat, obs\_lon)}, but you don't have to because
  the default values are Campbell Hall's values (which are {\tt lat$=37.873199$
  lon$=-122.2573$} degrees).

\end{enumerate}

\section{Your Second 21-cm Line Measurement (Group Lab Activity, Week 2)}
\label{meas2}

\noindent
Obtain a fully-calibrated spectrum for the horn pointing at Galactic
coordinates $(l,b)=(120^\circ, 0^\circ)$. How do you know where to point
the horn? You definitely do need to use rotation matrices!  See the
handout ``Spherical Coordinate Transformation''
(\S \ref{handouts}). Plot your spectrum versus both the observed and
the LSR velocity. 

\begin{figure}[h!]
\begin{center}
%\vspace{-0.7in}
%       
\includegraphics[scale=0.5]{bmp_cal1.pdf}
\end{center}
\vspace{-0.3in}
\caption{\footnotesize A typical raw spectrum with a curvy baseline and
  multiple velocity components. \label{rawspect}}
\end{figure}

Your spectrum will probably look something like that in the top panel of
Figure \ref{rawspect}. There is a zero offset, a curvy baseline, and
one or more spectral peaks. The curvy baseline comes from nonflat
instrumental response before the first mixer, and can be removed with a
low-order polynomial fit to the off-line channels (the approximate
frequency ranges $-1.2 \rightarrow -0.6$ and $ 0.6 \rightarrow 1.2$
MHz). The dashed line is a second-order polynomial fit to the off-line
channels and the bottom panel has this fit subtracted off.  Here, the
multiple peaks come from HI clouds having different velocities. The
peaks are usually represented as Gaussians with appropriate central
intensities, velocities, and widths.  Fitting Gaussians requires
specifying initial guesses for the component parameters. These guesses
need not be highly accurate. For example, in the bottom panel of Figure
\ref{rawspect}, initial guesses for the two heights, centers, and widths
(FWHM) could be [20, 50], [-0.3, 0.2] MHz, and [0.01, 0.03] MHz.  After
fitting, check to see if the result looks good: if it does not, then
it's usually because you need one or more additional components. It is
common to require a low-level, broad component that produces no obvious
peak; here, the initial guesses might be [20, 50, 10], [-0.3, 0.2, 0.]
MHz, and [0.01, 0.03, 0.07] MHz. 

The procedures you need for these least-squares fits
are: \begin{enumerate}
\item For the polynomial fit, use {\tt numpy.polyfit}

\item For fitting multiple Gaussians, use our home-grown {\tt gaussfit}; for
  evaluating the Gaussians from the parameters produced by {\tt gaussval}, 
  use {\tt gaussval}.
\end{enumerate}

\noindent
In class, we will compare the results from all groups. How reproducible is
our science?

\section {Measuring the Speed of Light with Waveguides (Group Lab Activity, Week 2)}
\label{expt}

\noindent
How do you move power or electronic information from one place to
another? {\it Transmission lines} (cables) and {\it waveguides} are
indispensable. It sounds easy---just connect two things with a wire. But
does this really work? We'll explore cables, waveguides, and reflections
by measuring the Voltage Standing Wave Ratio (VSWR), which results from
interference of incident and reflected waves.

The experimental work and measurements described below should
be done by groups.  The analysis in \S \ref{secondweek} should be done
by individuals. The analysis is nontrivial, so don't delay with the
measurements!


\subsection {The coax slotted line at C-band ($\sim 3$ GHz)} \label{slotted}

\noindent
Set up the slotted coaxial line with the wide-range HP 83712B
synthesizer.  With the far end of the slotted line open, find the
wavelength in the slotted line $\lambda_{sl}$ by measuring the positions
of {\it all} nulls as accurately as you can.  Do the same with the far
end shorted and note the differences.  In particular, note how the
positions of the nulls change when the slotted line is terminated with
an open versus when it is shorted.  Why is this?

From your measurements calculate the velocity [what does ``velocity''
  mean?] in the slotted line by comparing $\lambda_{sl}$ and the
frequency.  First do a quick calculation using your measurements. Then
do it as accurately as possible by using least-squares fitting and
following the advice in \S \ref{slottedls}.  {\it Note:} As it happens,
velocity is independent of frequency for a coax cable---in contrast
to a waveguide.  If you are so motivated, you can check this
experimentally.

\subsection {The X-band waveguide (7 to 12 GHz)} \label{waveguide}

\noindent
Now we'll do the same as we did in \S \ref{slotted}, but for X-band
waveguide.  The cutoff frequency for X-band waveguide is about 7
GHz. Pick a suitable frequency and, with the far end of the waveguide
completely open, measure the VSWR; then short the end of the waveguide,
maybe with a piece of aluminum foil or a metal plate, and repeat.  Why
is an open ended waveguide different from an open ended coax cable?

With the end shorted, measure the velocity in the slotted waveguide by
comparing the wavelength and the frequency; do this for several
(half-dozen?) frequencies, including one or two near cutoff, and compare
with theory.  These frequencies should be reasonably closely separated,
say by no more than 1 GHz.  From these measurements, derive the cutoff
wavelength (and thus the waveguide width) as accurately as you can.
Also measure the waveguide dimensions, and compare 
with the width you derived above.

\section {Statistical Error Analysis (Individually At Home, Week 2)} \label{secondweek}

\subsection{The coax slotted-line wavelength} \label{lbandcoax} 
\label{slottedls}

\noindent
For the measurements of \S \ref{slotted}, you sampled $M$
cycles of the standing wave in the slotted line. You can
calculate the wavelength $\lambda_{sl}$ from the distances
between a single null pair. While this calculation is a good
estimate, each of your measurements has an error. You have
measured the distance between $(M-1)$ null pairs. What's the
best way to combine these measurements so as to obtain the most
accurate $\lambda_{sl}$?

One way is to calculate the wavelength from each neighboring
null {\it pair} and take the average of the $(M-1)$ {\it pair}s.  You
might then apply the usual rule to obtain the uncertainty in the
average, i.e.\ the uncertainty is $\sigma/\sqrt{M-1}$.
However, this would {\it not be correct}.  Why? (Hint: The answer you
get by averaging the $(M-1)$ pairs is identical to that obtained from a
{\it single} well-chosen non-neighbor pair.  Which one?).  Actually,
this method gives {\it almost} the best estimate of the wavelength.  But
you can do better!

Alternately, use the fact that the positions of the nulls should increase linearly
with distance. So if $x_{m}$ is the position of null number $m$, we
should have

\begin{equation}
x_m = A + m {\lambda_{sl} \over 2} \ ,
\end{equation}

\noindent You know $x_m$ and $m$ from measurement and you would like to
derive $A$ and $\lambda_{sl}$. This is a classic least squares problem;
you're fitting a first-degree polynomial, where the unknowns are $A$ and
$\lambda_{sl}$. You can use {\tt numpy.polyfit}.

\subsection{The X-band waveguide}

\noindent
	RWvD provide a beautiful discussion of the TE$_{10}$ mode in
common rectangular waveguide (in which the ratio of side lengths equals
2).  Most important is their equation 11, which gives the {\bf guide
wavelength}, $\lambda_g$; we reproduce it here:

\begin{equation} \label{lambdaguide}
\lambda_g = {v_p \over f} = {\lambda_{fs} \over \left[ 1 - (\lambda_{fs}/2a)^2
\right]^{1/2} } \ ,
\end{equation}

\noindent where $\lambda_{fs}$ is the free-space wavelength.  The guide
wavelength is the wavelength within the waveguide, and combining it with
the frequency gives the propagation velocity in the guide. 

Here we want to do a least-squares fit of equation \ref{lambdaguide} to
solve for the waveguide width $a$. Now, you've already measured $a$ with
the caliper. But here the idea is to compare your measurement of the
$a$ with what theory (that's Equation \ref{lambdaguide})
predicts. So you imagine that $a$ is unknown and determine it from your
data on $\lambda_g$ versus the free-space wavelength $\lambda_{fs}$.
Doing this is not as straightforward as above in \S \ref{lbandcoax}
because $\lambda_g$ depends {\it nonlinearly} on $a$. The easiest way to
handle
this problem is by the brute force method, which we will discuss in
class. 

	From your above solution, determine the best-fit value of $a$
and compare it with the value you measured with the caliper.

\section {Reference Reading} \label{RWvD}

\noindent
You can do this lab without doing any reading, so this section, which
recommends parts of the highly readable and well-written text by Ramo,
Whinnery, and van Duzer (RWvD), {\it Fields and Waves in Communication
  Electronics}, is optional. If you have the time and inclination to
read, we suggest that you concentrate on the starred items below ({\bf
  ***}).  The notes below refer to the second edition, which should
reside in our lab's bookshelf.

\begin{itemize}
\item Theory of Coaxial Cable

\begin{itemize}
	
\item {\bf ***} {\it RWvD \S 5.2:  Introduction: what transmission lines do.}
Basic equations for transmission lines.  Note
equation (14) for the characteristic impedance $Z_0$, which is the ratio
of voltage to current in the line.  This is expressed in ohms, but it is
not ``ordinary resistance'' because voltage and current are out of
phase.  Example 5.2 calculates $Z_0$ for a coax line.  RG58/U, done in
the example, has 50 ohms; so do the cables we use in our system (in
contrast, the TV standard is 75 ohms).  Because of the dielectric in the
cable, the wave (phase) velocity is $\sim 0.7$ that in free space,
making the wavelength in the cable shorter than that in free space; one
needs to account for this when using cables to produce delays in a
signal, as in an interferometer or a quadrature circuit. 

\item {\bf ***} {\it RWvD \S 5.4: Reflection and transmission at a resistive
discontinuity.} The purpose of a transmission line is to take power from
a source to a load. The load should absorb the power and therefore
should be resistive. If its resistance matches the line impedance, then
all the power transmitted by the line is absorbed by the load. If not,
some is reflected; see their equations (4) and (6). We need to tune our
devices so that they are ``matched'' and reflect no power. 
\end{itemize}

\item The Theory of Waveguide
          
\begin{itemize}

\item {\bf ***} {\it RWvD \S 8.1:  Introduction: general description of
waveguides.} 

\item {\bf ***} {\it RWvD \S 8.8:  The $TE_{10}$ Mode in a Rectangular
Guide.} A nice, complete discussion of the TE$_{10}$ mode in common
rectangular waveguide (in which the ratio of side lengths equals 2). 
In this mode, E is perpendicular to the long side of the guide
(and also to the longitudinal axis of the guide).  We speak of the
polarization of the waveguide: the guide transmits linear polarization
that is perpendicular to the long side of a waveguide.  When we couple
power into a guide from coax, we do so with a probe that is parallel to
the internal electric field.  When we use a sliding probe to measure
VSWR in a waveguide, the probe extends through a long slot in the broad
side of the waveguide and the probe is parallel to the E field inside
the guide, so it samples the internal E field.  Note {\bf Figure 8.8},
which shows the current flow in the waveguide walls; note that the
current flow down the center of the broad side of the guide is parallel
to the axis of the guide---that is, parallel to the slot that we use for
the probe.  The slot, being parallel to the current flow, does not
interrupt any current flow and has no effect on the internal
distribution of fields in the guide.  Clever!

\item {\it RWvD \S 8.10.} Makes the analogy between coax cable and waveguide.
A coaxial line can be considered as a waveguide containing walls not
only the outside but also on the inside.  The usual mode that propagates
in coax is the TEM mode.  A TE$_{10}$-like waveguide mode enters for
wavelengths smaller than that given by their equation (4); at this
point, coax lines become basically unusable.  For type N connectors, we
are beginning to approach this limit in our 12 GHz system!

\item {\it RWvD \S 8.11.} Brief description of how coax lines are coupled to
waveguide. The method depends on which mode you wish to excite. For the
TE$_{10}$ mode, method (b) in Figure 8.11 works well and is almost
universally the one used.

\end{itemize}
\end{itemize}

\end{document}

%You can smooth over channels and reduce the noise. For example, suppose you
%want to reduce the the NSAMP channels in the power spectrum $pspect$ to 512
%channels. In IDL there are two
%easy ways to do this: \begin{enumerate}
%
%\item Use IDL's {\tt smooth} function: {\tt newspect = smooth(pspect,
%    NSAMP/512)}. This degrades the resolution, but you will still have
%  NSAMP channels in the output spectrum.
%
%\item Use IDL's {\tt rebin} function: {\tt newspect = rebin(pspect, 512)}.
%  This degrades the resolution and gives you 512 channels channels in the
%  output spectrum.
%\end{enumerate}

