\documentclass[11pt,preprint]{aastex}

\begin{document}
\def\simlt{\lower.5ex\hbox{$\; \buildrel < \over \sim \;$}}
\def\simgt{\lower.5ex\hbox{$\; \buildrel > \over \sim \;$}}

%\noindent Astronomy 121 \hfill 2004 Feb 10

\title {COMPLEX FOURIER TRANSFORMS AND SSB MIXERS \\
the week beginning March 18}

An earlier lab explored ordinary double-sideband (DSB) mixers, for which
the upper and lower sidebands produce identical mixer output. Here we
explore single-sideband (SSB) mixers, which display the remarkable
property that upper and lower sidebands produce outputs at negative and
positive baseband frequencies, respectively. In real life, SSB mixers
allow such things as the transmission and reception of stereo FM
signals; in olden days, FM transmission were all monophonic.

In that same lab we used discrete Fourier transforms (DFTs) to obtain
power spectra. We didn't look into the fundamentals of DFTs much---we
just plowed ahead. Here, it's time to study those fundamentals in some
detail from both the experimental and theoretical standpoints.

\section{GOALS}

\begin{itemize}

\item Learn how FFTs are a particular implementation of DFTs, how they
work, and what they provide.

\item Explore leakage power in DFTs.

\item Construct a two-output mixer, composed of two mixers, that can be
  operated as either a DSB or an SSB mixer. 

\item Use the two-output mixer in DSB and SSB modes and understand the
  difference. 

\item Learn how complex inputs to a FT break the negative/positive
  frequency degeneracy.

\end{itemize}



\section{ CONSTRUCT A TWO-OUTPUT MIXER USABLE FOR DSB OR SSB}
\label{upperlowerdsb}

From the block diagram given below in Figure \ref{ssbmfig} and in class,
construct a ``Computer Voodoo'' SSB mixer that achieves the phase delay
with a cable instead of a quadrature hybrid. We will use it to experiment
with no phase delay (a short cable) and a 90-degree phase delay (a long
cable).

For experimentation with this two-output mixer, use two SRS synthesizer
oscillators as inputs.  The SRS synthesizers work up to 30 MHz.  Assign
one of the SRS synthesizers to be your ``local oscillator'' with
frequency $f_{lo}$, and the other your ``signal'' with frequencies $f_{sig}
= f_{lo} \pm |\delta f|$.  To prevent aliasing and to filter out the sum
frequency, we will use the 5-MHz low-pass filters, one on each output of
the two-output mixer. Thus, we will want $|\delta f| < 5$ MHz. Moreover,
remember the Nyquist criterion: we want to see mixing products up to at
least 5 MHz, so you'll need to sample\dots at least how often?

The computer's ADC samples the two outputs simultaneously. We use the
two outputs as inputs to the DFT---one as the real, the other as the
imaginary input.

\section{THE DSB MIXER} \label{dsbmixer}

First see what happens when the phase delay cable is short, so that the two
outputs have no relative phase delay. Pick a value for $|\delta f|$ and
take time series data for the two corresponding values of $f_{sig}$ (these
are the upper and lower sidebands). Calculate the power spectra. When
taking the Fourier transform, be sure to make the inputs complex---you have
two simultaneous samples, one real and one imaginary. Looking at the power
spectra alone, can you distinguish between positive and negative $\delta
f$?


\section{THE SSB MIXER---EXPERIMENT}

We kind of lied above when we said that the quadrature hybrid is the
heart of the SSB mixer. The function of the first (top) hybrid is to
produce a $90^\circ$ phase shift between the two l.o.\ signals going to
the two mixers. We could generate this phase shift in another way:
replace the quadrature hybrid with an ordinary power splitter, and, in
one leg, a cable of appropriate length (which would, of course, be
$\lambda \over 4$).

Now see what happens when the phase delay cable introduces a relative phase
delay of $90^\circ$ between the l.o.\ signals going to the two mixers.
Repeat what you did above in \S \ref{dsbmixer}. Looking at the power
spectra alone, can you distinguish between positive and negative $\delta
f$?

If you have the time and inclination, verify that the phase difference
between the two mixer outputs behaves as shown in Figure \ref{dsbmixer}.
Why does it behave this way?  (Look at equations \ref{rhs} and \ref{lhs} in
\S \ref{mathdescr}.)

We could repeat the experiments on spectral leakage for the SSB case, but
we'd find the same kind of results as we did for DSB, so don't bother
unless you're really gung-ho.

\section{THE SSB MIXER---THEORY}

The SSB mixer has the capability of distinguishing whether the
difference frequency $|\delta f|$ is positive or negative---that
is, it distinguishes between the two sidebands.  The sidebands can, and
usually do, contain completely independent signals. 

\subsection{A Preliminary: The Quadrature Hybrid}

        The heart of a SSB mixer is not the mixer. Rather, it's the
quadrature hybrid. So before considering SSB mixers, we need to
understand hybrids and quadrature hybrids\footnote{For an excellent and
thorough discussion of hybrids and their applications, refer to:  Hagen,
Jon B., ``Radio-Frequency Electronics, Circuits and Applications,''
Cambridge University Press, 1996, pg. 188-200.}.


\begin{figure}[!h]
\begin{center}
$\begin{array}{cc}
        \includegraphics[width=2.0in]{hybrid1.plt}
        &
        \includegraphics[width=2.0in]{hybrid2.plt}
        \\
        \parbox[t]{3.0in}{\caption{Block diagram of a four port hybrid.
\label{hybrid1}}}\hspace{0.4in}        &
        \parbox[t]{3.0in}{\caption{Block diagram of a quadrature hybrid.
\label{hybrid2}}}
\end{array}$
\end{center}
\end{figure}


        Figure\ \ref{hybrid1} shows the block diagram of a hybrid.  The
signal acts just like the diagram shows: a signal entering port\ 1 will
split into equal signals (--3 dB each) and exit through ports\ 2 and 3.
No power is transmitted across the hybrid to port\ 4.  Internally, many
hybrids look like center-tapped transformers; see Hagen's book.

        Power is split between the two adjacent ports and nothing is
transmitted across it.  In this way we think of a hybrid as a {\it power
splitter}.  In fact, this is just what power splitters are: a hybrid
with the fourth port terminated in a matched load. If you are wondering
if it can be used in reverse as a power combiner, the answer is yes.  If
inputs are made to ports\ 1 and 4, the sum of the inputs exit through
ports\ 2 and 3. One point: all ports must present matched loads. By
symmetry, ports 2 and 3 each get half the power, so if one of these is
terminated then half the power is lost\footnote{Unless the combining
  signals are identical with a particular phase relationship, in which
  case all of the power can go to the unterminated port.}.

        Hybrids can be made so that they add a phase angle, either
$90^\circ$ symmetrically in two legs or $180^\circ$ in one leg.
Figure\ \ref{hybrid2} shows a block diagram.  A signal incident on port\
1
splits and exits through ports\ 2 and 3, but  each signal is
\emph{shifted} by the number of degrees specified in each corner of the
schematic:  the output of port\ 2 is shifted by $90$ degrees with
respect to the input while the
output of port\ 3 is not shifted at all.


\subsection{The SSB mixer: block diagram and graphical description}

\begin{figure}[ht]
        \begin{center}
        \leavevmode
        \includegraphics[width=3in]{ssbmix.plt}
        \end{center}
        \caption{Block diagram of the all-hardware and
                hardware/software combination SSB mixer. In the lab,
                we'll replace the upper quadrature hybrid with a
                $\lambda \over 4$ length of cable.}\label{ssbmfig}
\end{figure}

Figure\ \ref{ssbmfig} shows a block diagram of the SSB mixer.  The RF
input is split by a power splitter into two equal channels. The LO is
split by a hybrid, so that the left hand side (LHS) leads the RHS by
$90^\circ$.  These two channels are recombined with the second
quadrature hybrid.

Suppose that the signal ``RF in'' is a cosine wave in the {\it lower}
sideband, with frequency {\it below} the l.o.\ frequency. The l.o.\ is a
cosine wave on the RHS and a sine wave on the LHS (because of the
$90^\circ$ shift). This makes the two mixer outputs $\propto \sin
[\delta \omega t]$ and $\cos [\delta \omega t]$ for the LHS and RHS,
respectively. These are shown in Figure \ref{mixerout}, with the LHS
side [$MO(LHS)$] dashed and the RHS [$MO(RHS)$] solid. These are
identical in amplitude: they have to be, because they come from the same
original lower-sideband signal. But they are shifted in phase: for ``RF
in'' being in the {\it lower} sideband, the dashed curve (the left-hand
mixer) {\it lags} behind the solid one (the right-hand mixer).

Now suppose ``RF in'' is a cosine wave in the {\it upper} sideband, with
frequency {\it above} the l.o.\ frequency. Now the situation is
reversed: the dashed curve (the left-hand mixer) {\it leads} the solid
one (the right-hand mixer).


\begin{figure}[h]
%        \begin{center}
\hspace{-0.5in}
%        \leavevmode
        \includegraphics{ssbm.ps}
%        \end{center}
\caption{Outputs of the first mixers for the two sideband cases. Dashed
  curve shows the {\it left-hand} mixer, solid is the {\it right-hand}
  mixer. Left panel shows ``RF in'' being in the {\it upper} sideband
  (USB); right panel shows it being in the {\it lower} sideband (LSB).
\label{mixerout}}
\end{figure}

\subsection{The Digital ``Computer Voodoo'' Method}

In Figure \ref{ssbmfig}, we can regard the two independent signals
marked ``To ADC for Computer Voodoo'', as as the real and imaginary
components of the signal because, no matter what the input frequency,
they differ in phase by $90^\circ$. So the signal is a complex
number. With an ordinary DSB mixer the signal is a real number, so its
Fourier transform must have Hermitian symmetry---meaning that the power
spectrum is symmetric, with negative and positive frequencies being
identical. Here, the signal is a complex number, and there is no
symmetry. We won't delve into the math here, but the net result is that
the Fourier transform keeps the negative and positive frequency
components completely separate. Real geeks call this method ``QU
sampling''. 

\subsection{Mathematical description} \label{mathdescr}

First, we mix the incoming signal with the real and imaginary (cosine
and sine) components of the l.o. As in the DSB mixer, we use a low pass
filter so we need consider only the difference frequencies. For the left
hand side ($LHS$) the l.o.\ has the $90^\circ$ phase difference so it's
a $\sin$ instead of a $\cos$ and for the Left-Hand Mixer Output
[$MO(LHS)$] we have

\begin{equation}
MO(LHS)_\pm = {E_s \over 2} \sin [\pm|\delta \omega | t] \ .
\end{equation}

\noindent {\it Note} the -- and + subscript on $MO(LHS)$: these refer to
``RF in'' being in the {\it lower} and {\it upper} sidebands (LSB and
USB), respectively. 

For the LSB input we have

\begin{mathletters} \label{lsb}
\begin{equation}
MO(LHS)_- = {E_s \over 2} \sin [-|\delta \omega | t] =
   -{E_s \over 2} \sin [|\delta \omega | t]
\end{equation}
\begin{equation}
MO(RHS)_- = {E_s \over 2} \cos [-|\delta \omega | t] =
   {E_s \over 2} \cos [|\delta \omega | t]
\end{equation}
\end{mathletters}

\noindent Similarly, for the USB input we have

\begin{mathletters} \label{usb}
\begin{equation}
MO(LHS)_+ = {E_s \over 2} \sin [+|\delta \omega | t] =
   {E_s \over 2} \sin [|\delta \omega | t]
\end{equation}
\begin{equation}
MO(RHS)_+ = {E_s \over 2} \cos [+|\delta \omega | t] =
   {E_s \over 2} \cos [|\delta \omega | t]
\end{equation}
\end{mathletters}

\noindent The RHS and LHS outputs have identical frequencies and
amplitudes for the LSB and USB inputs. 

{\it But the phase relationships differ for the two sidebands.}  For the
{\it lower} sideband, the RHS output {\it lags} the LHS by $90^\circ$;
for the {\it upper} sideband, the RHS output {\it leads} the LHS by
$90^\circ$, just as in Figure \ref{mixerout}. In other words, for the
LHS output, lower-sideband signals differ in sign from upper-sideband
ones; this is equivalent to differing in phase by $180^\circ$.

When we sample these two mixers simultaneously, and use one mixer as the
{\it real} input to the DFT and the other as the {\it imaginary} input,
the output frequency spectrum {\it separates the sidebands}. That's why
this is known as a Single-Sideband Mixer, or Sideband-Separating
Mixer---a SSB. 

To see this, use equation \ref{**}, writing $\delta \omega$ instead of
$2 \pi \nu$.  For the LSB in equations \ref{lsb}, the two mixer outputs
are ${E_s \over 2} \sin(-|\delta \omega|t)$ and ${E_s \over 2}
\cos(-|\delta \omega|t)$; using these as the imaginary and real inputs
to fhe Fourier transform of equation \ref{***} and applying the trig
identities, the integrand becomes

\begin{equation}
{\rm integrand} = {E_s \over 2} \left( \cos[(\delta \omega +
    \omega)t] + j \sin[(\delta \omega + \omega)t] \right) \ .
\end{equation}

\noindent For large $T$ in equation \ref{***}, the integrand is nonzero
only when $[(\delta \omega + \omega)t] =0$. For the LSB with negative
$\delta \omega$, the integral is nonzero only for positive $\omega$.

\end{document} 

