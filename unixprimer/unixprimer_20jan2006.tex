\documentstyle[11pt,aaspp4]{article}
%\documentstyle[11pt,/home/ay120b/tex/aaspp4,/home/ay120b/tex/epsf,/home/ay120b/tex/eqsecnum]{article}
\pagestyle{plain}
\begin{document}

\setlength{\parskip}{.02in}

\title{A SHORT UNIX PRIMER}

\author{Carl Heiles \altaffilmark{1} \\ \today}

\altaffiltext{1}{Based on earlier versions by Anthony Truong and
Jennefer King.}

\tableofcontents

\section{BEFORE WE BEGIN}\label{intro}

	This handout is in two basic sections.  The first (\S 2) is for
the first-time user, who has either no computer experience whatsoever or
who has worked only with Windows (which is almost equivalent to having
no computer experience whatsoever).  The second (\S 3) contains
information that many people might find handy.  Finally, \S
\ref{commands} is a list of basic UNIX commands that everybody ought to
know for working efficiently. 

\section{FOR THE FIRST-TIME USER ONLY} \label{first}

	The best way, by far, to learn Unix is to sit in front of a
computer for an hour or so and to try it out yourself.  Once you master
some basic commands, you'll be amazed at the many useful things you can
do. 

\subsection{Logging In}\label{login}

	Our Sun computers use an operating system called Solaris.  Under
Solaris, you have a choice of desktop environments: {\it Open Windows}
and the {\it Common Desktop Environment (CDE)}, which is the default. 
Open Windows is easier to use and takes up less memory.  To use Open
Windows, you must, before you log in, change the session to Open Windows
under the Options menu. 

\subsection{Customizing the DeskTop}
	You can customize details of your desktop such as colors and
fonts.  To accomplish this, first place the cursor on the background and
right-click the desktop; then click {\it properties.  } A window should
pop up allowing you to change the colors for your viewing pleasure.  By
clicking {\it Category}, you can change all sorts of things other than
color, too.  

	As you can see by now, by {\it right}-clicking the desktop, a
menu with a lot of useful commands will appear.  Experiment!

\subsection{Directory Structure}\label{dirstruct}

	The directory structure of Unix is organized in a hierarchical
system, or by levels.  Each user's top level is the home directory in
which you have full read/write/execute access.  When you move throughout
the directory system, you are moving from {\it level to level}.  This
directory is called {\em /home/loginname}.  Figure 1 shows what a sample
directory tree structure might look like.  Here, {\it home}, {\it
loginname}, and {\it lab1} are directories.  Notice that these
directories are nested within one another; thus {\it loginname} is a
subdirectory of {\it home}, and {\it lab1} is a subdirectory of {\it
loginname}. 

\begin{center}
\begin{figure}[h!]
\epsfysize=3in
\epsffile{dirtree.ps}
\caption{A sample Unix directory tree.}
\end{figure}
\end{center}

	Organizing your files in subdirectories is very important
because over the semester you will be generating hundreds, if not
thousands, of files.  It makes sense to keep all files relevant to the
first lab in a special subdirectory for that lab, called say {\it lab1},
and perhaps you might wish to have separate subdirectories under {\it
lab1} for the data, the writeup, and the analysis. 

	And specifying the {\it permissions} of accessibility is also
very important.  You might want the data to be accessible by everybody,
both for reading and writing, but you certainly wouldn't want the
writeup to be accessible by everybody, especially for writing.  And it
makes sense to keep all your love letters in a separate file that isn't
accessible in any way by anybody else---including their reicipent(s)!

\subsection{Working with Directories}\label{dir}
	
	First, check to see where you are.  At the Unix prompt, type the
following:

\noindent {\bf ttauri\% pwd}

\noindent The command {\bf pwd} stands for {\it present working
directory}.  The computer should return something along these lines:
{\bf /home/loginname}.  From this you know that you are in your {\it
loginname} directory, which is located {\it beneath} the {\it home}
directory. 

	To see a listing on what already exists in your directory, use
the {\bf ls} command.  This gives you a listing of the subdirectories
and files in the directory that you are currently in. 

\noindent {\bf ttauri\% ls}

\noindent Depending on what's in your directory, you'll see different
things.  In general, directories are indicated with a ``/'', as in {\it
dirname}/, while files usually are just listed as {\it filename}.  If
you've just gotten your account, chances are there's not much in your
home directory.  As you work with Unix more and more, it will balloon to
monstrous proportions unless you create subdirectories and place the
file appropriately. 

	Let's create a subdirectory called {\it lab1} that resides just
under the home directory. To accomplish this:

\begin{tabbing}
\textbf{cd $\sim$} \hspace{1in} \= Make sure you are in your home
directory (see below). \\
\textbf{mkdir lab1} \> create the directory {\em /home/loginname/lab1}
\\
\end{tabbing}

\noindent Now check to see the contents of your directory, using the
{\bf ls} command.  Note that {\it lab1} is listed as a directory,
denoted as {\it lab1}/.  Also notice something else---the directory that
you've created is located in the directory that you were in when you
typed the {\it mkdir} command.  Make sure you know where you are when
you create a new directory or file.  (That's why {\it pwd} can be so
useful.)

	To move into and out of directories, use the {\bf cd} command. 
The command used to move into a directory {\it below} your current
location is {\bf cd dirname}.  To move {\it up} from your present
directory one level, enter at the prompt {\bf cd ..} instead.  So if you
were in your {\it loginname} directory, to move up into the {\it home}
directory you would enter {\bf cd ..} at the prompt, and to move down
into your {\it lab1} directory you would enter {\bf cd lab1}.  You can
also move into directories by specifying the entire path---so if you
wanted to get to your {\it lab1} directory you could type {\bf cd
/home/loginname/lab1}. 

	The \textbf{$\sim$} character is a shortcut to your home
directory.  Suppose you are at some arbitrary subdirectory, deep down in
the structure; to get to your home directory, you'd type \textbf{cd
/home/loginname}.  But the shortcut allows the much easier \textbf{cd
$\sim$}.  You can also use it as a shortcut to accessing your
subdirectories.  Suppose you want to get into your newly created lab1
subdirectory, then you can type either \textbf{cd /home/loginname/lab1}
or \textbf{cd $\sim$/lab1}.  Of course, if you know that you're already
in your home directory, you can just type {\bf cd lab1}. 

	You can delete (remove) a directory (say, {\it lab1}) by first,
deleting (removing) all the files in the directory (see below), and then
typing

\noindent {\bf rmdir lab1}

\subsection{Creating and Viewing Files}\label{files}

	Files can be created by using a text editor; it allows you to
create a file and type in whatever you want.  You should
learn {\bf emacs}.  Suppose you wish to compose a love letter and call
the file {\it love1}.  
	
\noindent {\bf ttauri\% emacs love1}

A new x window will pop up with a bar on the top that has menu-driven
commands. You can experiment with these. But you should also learn the
basic keystroke commands because they work not only in emacs but also on
the unix command line. We summarize them below in \S \ref{keycommands}. 

	Suppose you want to quickly view this file without editing it:
type either {\bf more love1} or {\bf less love1} [which would you rather
have? in this case, less is more! (because the arrow keys work as you'd
expect and you can more not only downwards but upwards)].  The first
page of your file will appear, and you can either view the rest of the
file a page at a time by hitting the space bar, or line by line using
the Return button.  To stop looking at the file, hit {\bf q} or {\bf
Ctrl-c} and you will be returned to the Unix prompt. 

\subsection{Moving and Copying Files}\label{movecopy}

	Two other commands that you'll find useful will be the copy and
move commands.  The copy command takes an existing file and copies its
contents into a new file.  The move command takes an existing file and
moves it to a new file\footnote{Note the subtle difference here---using
copy, you are left with two files, with move only one file remains.}. 

\noindent {\bf ttauri\% cp love1 love2}

\noindent Here the contents of {\it love1} are copied into a new file
named {\it love2}; this is handy if you want to send the same letter to
a different person, for example, because you can edit {\it love2} to
change the name.  If you want to copy it to a different directory, just
include the directory name as part of the filename, like

\noindent \textbf {ttauri\% cp love1 $\sim$/lab1/love2}

\noindent You can {\it always} specify the filename with a different
directory name in this way. Similarly, you can move a file with

\noindent \textbf {ttauri\% mv $\sim$/lab1/love2 $\sim$/lab1/tex/love3}

\subsection{Removing Stuff... Yikes!}\label{remove}

	So far you've learned only how to create files and directories,
but what about removing them? This can be done just as easily.  The
command to remove a file is simply {\bf rm filename}.  The {\bf rm}
command has options, one of which is to force you to confirm that you
{\it really} want to remove the file; we include this option when we set
up your default login files for new accounts.  Be sure that you want to
remove a file when you do this! Once it's been removed, it's gone for
good.  The same goes for removing a directory.  The command here is {\bf
rmdir dirname}.  Always exercise caution when removing anything. You
really should define an aliases for the following commands 
in your {\it .aliasfile} by
entering the lines

\noindent {\bf alias rm "rm -i"}

\noindent {\bf alias mv "mv -i"}

\noindent {\bf alias cp "cp -ip"}

\noindent these will force the system to ask you permission before
overwriting anything\footnote{But in UNIX some commands still have the
power to overwrite inadvertantly.  You'll learn this much to your
chagrin some day.  Be sure to keep {\it current copies} of anything
important in a different directory.}
	
	Example: if you wanted to remove the file {\it love1}, first
you'd get into the directory that contains that file, and then you would
enter:

\noindent {\bf ttauri\% rm love1}

\noindent Suppose that directory was {\it lab1} and you also wish to
remove the whole directory.  First, remove all the files from that
directory; then to remove the {\it lab1} directory, move into the
directory just above it and then type:

\noindent {\bf ttauri\% rmdir lab1} 

\section{FOR EXPERIENCED USERS}

\subsection{More About Listing Files, or the {\it ls} Command}

	If you want to see a listing of the files in your directory, an
{\bf ls} will give you all you need to know.  But sometimes you have
other questions too, such as: How big are my files? What's the last file
I worked on? This is where options can come in handy.  Options can be
included in an {\bf ls} command to enable your files to be listed in a
way that's the most useful to you.  Let's say you wanted to know what
the last file is that you worked on.  You can use the {\bf -t} option, and
enter at the prompt {\bf ls -lt}.  This should give you a listing of
your files from most recently accessed to least recently accessed; the
command {\bf ls -lrt} gives the same, but in reverse order.  The
procedure for using other options is the same.  The following is a list of
some of the options you can use with the {\it ls} command:

\begin{tabbing}
\textit{-a} \hspace{0.5in} \= Lists all of your files, including the dot, or hidden ones. \\
\textit{-1} \> Your files are listed in a single column output. \\
\textit{-s} \> The sizes of your files are listed. \\
\textit{-t} \> Lists your files from most recently accessed to least recently accessed. \\
\textit{-l} \> Long format; meaning that info such as permissions, owner, size, and date are included. \\
\end{tabbing}

	These are just a few of the options available.  To get more
info, use {\bf man ls}.  You can get more info on {\it any} command by
typing in {\bf man commandname}.  Usually there'll be more info than
you'll ever need on that specific command; but it's definitely a good
resource to keep in mind. 

\subsection{Remote Logging In}\label{remote}

	You can log in from home {\it if} your computer has the secure
login software, called {\it ssh2}. 

\noindent {\bf ssh2 rrlyrae}

\noindent If you're logging in from home on an (ugh!) windows machine,
you need to be able to use X windows; for this, run the program {\bf
exceed} before logging in. You can get such software  from the CD  {\it
Connecting @ Berkeley}, available for free from the big U.

\subsection{Aliases and Shortcuts}\label{short}
	
	Aliases are shortcuts that you can use in place of typing out a
long command over and over again.  You can define an alias on the
command line; alternatively, if you want to define it permanently, you
can define it by editing your {\it .cshrc} file (which resides in your
home directory). 

	Here's an example: suppose you want to force UNIX to check
whether it will overwrite a file when you use the {\bf mv} command and,
also, to ask you about it. To do this,
you use the {\bf -i} option, and you redefine the command {\bf mv} by
typing

\noindent {\bf alias mv "mv -i"}

\noindent If you type it in a window, it will apply henceforth to that
window alone.  If you include it as a line in your {\it .cshrc} file, it
will always apply (after you login again).  You can check the definition
of an alias by typing

\noindent {\bf which mv}

\subsection{Permissions}\label{perm}

	Permissions are important for security and privacy reasons and
are set with {\bf chmod} and its options.  Among the users, there are
three groups: {\bf u}user (yourself), {\bf g}roup (undefined here),
world (``{\bf o}ther'') ({\bf u, g, o}).  Among the permissions, there
are three important ones: {\bf r}ead, {\bf w}rite, e{\bf x}ecute ({\bf
r,w,x}).  {\bf chmod} allows you to add, take away, and set exactly
permissions for different users with the operators ({\bf +, --, =}). 

	For example, to restrict your love letter {\bf love1} from being
read, executed, or written on by everybody but yourself, type

\noindent {\bf chmod go--wxr love1} \hspace{0.5in} (from group and other,
remove write, execute, and read)

\noindent You can apply permissions to a whole subdirectory instead of
single file.  For example, your datafiles are located in
\textit{$\sim$/lab1/data} and you wish to allow the world to read all
your datafiles but not to write on them, so you type

\noindent \textbf {chmod go+r,go--wx $\sim$/lab1/data}

	You can check permissions with the {\bf ls -l} command, for
which the permissions are listed in the first 9 columns in three groups
of three columns.  The first group of three columns is the individual
user, the second the group, and the third the world; and within each
group, the first is execute, the second read, the third write.  Try it!

\subsection{X Windows, Command-Line Editing, and Emacs}

\label{keycommands}

	You should always use X windows in UNIX because you can then
both scroll backwards and use command-line editing.  The editor of
choice is {\bf emacs}, for several reasons; among these is that it
incorporates the same editing commands that work for command-line
editing in X windows.  The most important editing commands are:

\begin {tabbing}
{\bf arrow keys} \hspace{0.8in} \= move the cursor as you'd expect. \\
{\bf Ctrl-d} \> deletes the character under the cursor. \\
{\bf backspace} \> deletes the character behind the cursor. \\
{\bf Ctrl-e} \> moves the cursor to the end of the line. \\
{\bf Ctrl-a} \> moves the cursor to the beginning of the line. \\
{\bf Ctrl-k} \> deletes the the rest of the line.
\end{tabbing}

\noindent Sometimes, when command-line editing, you inadvertently hit
{\bf Ctrl-s}; this prevents the cursor from responding to your
keystrokes. If you encounter this condition, type {\bf Ctrl-q}, after
which things will work normally again.

	In Solaris, you can customize any X window to your desires
(e.g.\ fontsize) by putting the cursor on the window, holding down the
CTRL key and, simultaneously, holding down a mouse button; each one
provides different options.  If you want to customize them permanently,
edit the {\it .openwin-init} file in your home directory. 

\subsection{Piping, etc: $|, >, <$}\label{help}

	Piping directs the output of a command to the next succeeding
command.  For example,

\noindent \textbf{ ls $|$ grep /}

\noindent directs the output of the listing command to {\it grep}, which
here selects all names containing the string ``/''; those are
directories, so this gives a list of directories just under the current
directory. 

	Normally the result of a UNIX command is written to the terminal
for you to see. However, you can direct the output elsewhere. For
example, 

\noindent \textbf{ cat love1 love2}

\noindent writes the concatenation of the files {\it love1} and {\it
love2} onto the screen, while

\noindent \textbf{ cat love1 love2 $>$ loveboth}

\noindent writes the concatenation into a new file called {\it
loveboth}.

	Normally the input to a UNIX command is expected to be from the
terminal. However, you can get the input from elsewhere. For example, 

\noindent \textbf{ mail heiles $<$ complaint.txt}

\noindent mails the file {\it complaint.txt} to heiles; try it! (Do you
think it will do any good???)

\eject

\section{COMMON COMMANDS}\label{commands}

	Following is a list of some useful UNIX commands, which can be
used as a quick reference.  
 

\begin{tabbing}
\textit{passwd} \hspace{1.6in} \=  Useful for changing your account password. \\
\textit{man commandname} \> Gets info for you on a specific command. \\
\textit{pwd} \> Shows your ``present working directory''. \\
\textit{cd dirname} \> Moves you into the subdirectory, below your 
present directory. \\
\textit{cd ..} \> Moves you out of a subdirectory into the directory above
it. \\
\textit{mkdir dirname} \> Creates a subdirectory named \textit{dirname}. \\
\textit{rmdir dirname} \> Removes a subdirectory named \textit{dirname}. \\
\textit{rm filename} \> Removes a file named \textit{filename}. \\ 
\textit{cp file1 file2} \> Copies the contents of \textit{file1} into 
\textit{file2}. You are left with two files. \\
\textit{mv oldfile newfile} \> Moves (or renames) \textit{oldfile} as 
\textit{newfile}. \\
\textit{cat file1 file2 $>$ fileboth} \> Concatenates {\it file1} and {\it
file2}, writing them into the new {\it fileboth}. \\
\textit{which cp} \> tells the current definition of {\it cp} ({\it which}
works for any command); \\
\textit{history} \> gives a numbered list of the previous commands you've
typed; \\
\> typing {\bf !number} repeats that command. \\
\textit{!!} \> Repeats the previous UNIX command. \\
\textit{find dirname -name filename} \> finds all files with {\it
filename} in and under the directory with {\it dirname}. \\
\textit{find dirname -name '*love*'} \> Finds all files whose names 
contain the string ``love''. \\
\textit{ls -lrt} \> Lists the files and subdirectories 
in the present directory. \\
\>The {\it -lrt} gives a long format in reverse time order.\\
\textit{ls -lrt $|$ grep /} \> Pipes the output to {\it grep}, which \\
\> selects only those names containing ``/'' (which are directories). \\

\textit{du -k dirname} \> Tells kilobytes used by everything in dirname. \\
\>Also handy for giving the directory tree structure. \\

\textit{df -k} \> Tells kilobytes used and available on all disks. \\


\textit{grep -il text file} \> Searches the \textit{file} for occurrences of
the string \textit{text} \\
\> The {\it -i} ignores capitalization and {\it l} lists only the filename.\\
\textit{lp filename} \> Prints the file \textit{filename}. \\
\textit{less filename} \> Shows you the contents of the file named 
\textit{filename} one screen at a time; \\
\> more flexible than {\textit more}. \\
\textit{tail -40 filename} \> Shows you the last 40 lines of the file
{\it filename}. \\
\textit{lpq} \> Displays the print queue. \\
\textit{cancel jobnum} \> Removes the \textit{jobnum} in the print
queue. You must own the job \\

\textit{top} \> Shows CPU usage, etc, for jobs on your machine. \\
\textit{ps -u username} \> List the programs that \textit{username} is
currently \\
\>  running on the machine you are logged onto. \\
\textit{kill processnum} \> Kills the process listed with
\textit{processnum}. You must own the process \\
\end{tabbing}
	



\end{document}


