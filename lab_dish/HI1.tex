\documentclass[preprint]{aastex}
\usepackage{graphicx}
\begin{document}
\tolerance=10000
\def\simlt{\lower.5ex\hbox{$\; \buildrel < \over \sim \;$}}
\def\simgt{\lower.5ex\hbox{$\; \buildrel > \over \sim \;$}}

\title{THE HI LINE, THE GALAXY, and SUPERNOVA SHOCKS}

\tableofcontents

\section{INTRODUCTION}

	Diffuse atomic gas permeates all corners of interstellar space.
On the largest scales, its distribution reveals a lot about Galactic
structure and dynamics. On smaller scales, its morphology is dominated
by shocks from energetic stellar winds and supernovae. On still smaller
scales, magnetic fields, pressure differences, and thermal instability
shape the gas into thin filaments and sheets. On still smaller scales,
the densities get large from self-gravitation; the gas turns molecular
and stars form. 

For this lab, we outline a several projects in HI that are suitable for
our 4.5-m diameter Leuschner dish. They fall into two categories. One
explores large-scale Galactic structure and dynamics by obtaining HI
spectra along great circles, presenting the data as position-velocity
images, and interpreting those maps as discussed below. The other maps
specific objects by sampling a grid of spectra over a large area,
presenting the data as grey-scale images of velocity-integrated line
intensity (i.e., antenna temperature $T_A$) or as images in which image
brightness indicates velocity-integrated $T_A$ and color indicates
velocity, and interpreting the data.

We'd like each group to pick a single project and carry it through to
completion by the time the lab reports are due, which is the last day of
instruction. All projects involve making maps and images. One of
the features of this lab is to learn about image processing and using
colors in images.

For the great-circle projects (\S \ref{galacticplane} and \S
\ref{galacticpoles}), we gain large angular scale information
in the sky by observing on great circles. Getting nice results
requires data over the largest portions of the great circles you can
get. For a full 360-degree circle, that's 180 pointings at $2^\circ$
spacings.  Mostly, though, we can't cover a full great circle because
some of the circle lies too far south, so we do as much as we can. 

\section{SCHEDULE}

The end of the semester is approaching like a freight train. Imagine
yourself tied onto the tracks by a madman: you've gotta finish before it
hits. Here's a schedule. If you don't keep up with this, you're in
trouble! \begin{enumerate}

\item Week 1: Get the standard spectrum with the big horn
  on Campbell's roof.

\item Week 2: Pick a group project. Get the standard
  spectrum with the Leuschner dish. Figure out how you will get the data
  for your project. Get some data. Reduce the data.

\item Week 3: Get the rest of your data. Figure out
  how you want to present the data as images. Make some first-try
  images. 

\item Week 4: Finalize data. Make final images. Write
report. Report is due the last day of claas, 03may.
\end{enumerate}


\section{ON MAPPING AREAS: HOW TO SPATIALLY SAMPLE} \label{sampling} 

All projects examine the angular dependence of the 21-cm line profile,
so you need closely-spaced observations to make a map or an image. How
closely should the observations be spaced?  The usual criterion is two
samples per HPBW (Half-Power BeamWidth), because observing at more
closely spaced angles just gives redundant information and takes extra
time, while more widely spaced angles loses small-scale structural
information. (This statement can be justified more exactly by Fourier
transforming the structure of the sky in terms of angular frequency and
then imposing the Nyquist criterion\dots).  The HPBW is $\sim 4^\circ$;
using a spacing of $2^\circ$ is close to optimum, and it's being a round
number is nice.

For the mapping projects (not the great-circle ones), we need to cover
areas of 1000 to 2000 square degrees. Consider the Orion-Eridanus
Superbubble (\S \ref{eridanus} below), which wants to map the the region
covering roughly $(\ell=160^\circ \rightarrow 220^\circ, b=-70^\circ
\rightarrow -10^\circ)$. You might think you'd set up a regular grid in
$(\ell,b)$ with $(\Delta \ell, \Delta b)$ both equal to
$2^\circ$. That's find for $\Delta b$, but you'd be sampling too closely
in $\ell$. That's because the true (great-circle) angular distance
between two points is $\Delta \ell \cdot \cos(b)$; there's a
foreshortening for $b \ne 0^\circ$ (e.g.: when you're at the Earth's
North Pole, you can go completely around a small circle of constant $b$
with just a few steps.). So you can save lots of observing time by
observing at $\Delta \ell = {2^\circ \over \cos(b)}$ instead of $\Delta
\ell = 2^\circ$.

The true angular area is not the total extent in $\ell$ times the total
extent in $b$ because of the foreshortening discussed in the above
paragraph. So the actual angular area of the Orion-Eridanus area area
isn't the range in $\ell$ times the range in $b$, i.e.\ $60^\circ
\times 60^\circ = 3600$ square degrees (which would require 900
pointings), but rather only about 2800 square degrees.  At $2^\circ$
sampling, that's about 700 pointings.

These projects require what seems like a lot of pointings,
but---hey!---the telescope is automated. With the magic of our
computer-controlled telescope you can do these observations in just one
or two days. But it will require planning and coordination with other
groups---all of you are in the same boat! (Or should we say, ``in the
same dish''?) I suggest some trial observations to get started, both
making sure you are observing correctly and also getting your datataking
and reduction software in shape.

\section{THE GALACTIC PLANE: THE GREAT CIRCLE AT LATITUDE ZERO 
\boldmath{$(b=0^\circ$)}} 
\label{galacticplane}

	Point the telescope at a series of Galactic longitudes ($\ell$)
along the Galactic plane (Galactic latitude $b = 0^\circ$), over the
full longitude range that is observable for our telescope.  Getting nice
results requires data over the largest section of the Galactic plane you
can get, which is roughly (or somewhat less than) $\sim -10^\circ \simlt
\ell \simlt 250^\circ$. This observation has two science goals:
\begin{enumerate}

	\item {\it The Galactic rotation curve}. Determine the rotation
curve of the Galaxy for the portion of the Galaxy inside the Sun and use
the result to estimate the {\it gravitational} mass $M_{grav}$ of the
Galaxy that lies inside the Solar circle; for this, you just assume the
usual $v^2 = {G M_{grav} \over R}$.  In fact, you can even get the
radial dependence $M_{grav}(R)$! Also, use your observations to estimate
the {\it gaseous} mass $M_{gas}(R)$ of the Galaxy.  What fraction of the
total Galactic mass comes from interstellar gas? For this you'll need to
know the radius of the Solar circle: it's 8.5 kpc.  See {\it Commentary}
below for a discussion. 

	\item {\it Spiral Structure of the Galaxy}.  All your life,
you've been told that we live in a spiral galaxy.  Are they lying? In
principle, to detect spiral structure you make a position-velocity plot
along the Galactic plane [that's the image $T_A(\ell,V_{LSR})$ for
$b=0^\circ$].  Bright regions are regions of excess density---or regions in
which lots of HI is packed into a narrow velocity range (see \S
\ref{coldensity} below). 

	Do you see spirals in your position-velocity plot? How do you
tell? There's only one easy way.  Make a model of a spiral arm and see
how it projects onto position-velocity space.  To do this, you need to
know the rotation curve.  You've measured it for the inner Galaxy; for
the outer Galaxy (outside the Solar circle) assume that the rotation
curve is constant beyond the solar circle [$V(R)$=220 km/s for $R > 8.5$
kpc].  In polar coordinates, the equation for a spiral arm is

\begin{equation}
R_{arm}  =  R_0 e^ {\kappa (\phi - \phi_0)} 
\end{equation}

\noindent where $R_0$, $\kappa$, and $\phi_0$ are free parameters;
$\kappa$ is the tangent of the {\it pitch angle}.  You project this into
position-velocity space using equation \ref{vdopp} and your knowledge of
the Galactic geometry.  

{\it Comment:} Finding spiral arms inside the Solar circle is very
difficult, and has thwarted efforts by astronomers over the past several
decades. Finding them in the outer Galaxy is not so bad.

\end{enumerate}

\section{IS THE GALACTIC PLANE FLAT?}

	External galaxies are often distorted (like the brim of a Fedora). 
These distortions are probably produced primarily by nearby neighbors. 
How about our own Galaxy? To determine this we want to look at the
vertical structure of the Galactic plane and see where the peaks occur;
a flat Galaxy would everywhere have the peaks at $b=0^\circ$. 

The most comprehensive way to observe and display these distortions is
with a map that covers the biggest possible swath in Galactic longitude
$\ell$ for the range of Galactic latitude ($b$) from $-20^\circ$ to
$+20^\circ$. This is a huge area---almost $10^4$ square degrees! At
$2^\circ$ sampling, you'd need $\sim 2500$ profiles.  To make the
project more manageable, you can increase the increment in $\ell$, say
to 4 degrees, while keeping the same, fully-sampled increment in
$b$. Resolution is $l$ is less important because---like a
fedora---the warp doesn't change rapidly with $l$.

This project affords a great opportunity to make a spectacular color
image in $(\ell,b)$ coordinates: brightness showing the amount of gas,
color the Galactic rotation.  Officially, the science goals here are to
estimate the thicknesses of the Galactic disk and, also, to obtain an
approximate Fourier representation of the warp---which is not so
straightforward given the absence of data in the ``Southern'' Galaxy.

\section{THE GALACTIC POLES: GREAT CIRCLES AT CONSTANT LONGITUDE $\ell$}
\label{galacticpoles}

How is the interstellar gas distributed within the Galaxy---for example,
is it ``disklike'' or more spherically distributed? Are there any
systematic motions other than just ``Galactic rotation''? (You'd be
surprised!).  To answer these questions, determine the column density
and mean velocity of interstellar gas as a function of Galactic latitude
$b$. 

	To answer this completely you'd need to do a sky survey! That's
a big job.  So let's do something easier. We have one project (\S
\ref{galacticplane}) that looks around a great circle, the Galactic
equator (constant $b$). Other projects could check out one or more
orthogonal great circles (at constant longitudes $\ell$) going from one
Galactic pole to the other and back again (or, if not the whole 360
degree circle, as much as possible).  Particularly appropriate sets of
constant-longitude great circles are the pairs $\ell = (220^\circ,
40^\circ)$ (partly because you can get the whole great circle) and $\ell
= (130^\circ, 310^\circ)$ (because at positive latitudes there's weak
high-velocity gas).

	The science goals here are to characterize the thickness of the
gas layer and to determine the vertical kinematics of the gas, and if
possible an estimate of the energy involved in the vertical motions.

%\section{MAPPING THE NORTH CELESTIAL POLE} \label{celestial}
%
%{\bf NOOOOOOOOOOOOOOOOOOOOOOOOOOOOOOOOOOOOOO!!!!!}
%The region near the North Celestial pole contains a large shell,
%probably produced by one or more supernovae, and also to has angular
%scales that are well-matched to our telescope and the available time for
%your project. For this, you'd map the region covering roughly
%$(\ell=105^\circ \ {\rm to} \ 160^\circ), (b=15^\circ \ {\rm to} \
%50^\circ)$. Interesting features should produce antenna temperatures
%$\sim 10$ K. This is about 1600 square degrees and, with 2 degree
%spacing, requires about 400 profiles. 

\section{MAPPING THE ORION-ERIDANUS SUPERBUBBLE} \label{eridanus}

The ``Orion-Eridanus Superbubble'' was, and continues to be, produced by
energetic stellar winds and supernovae that were located in immediate
vicinity of the Orion nebula. For this, you'd map the region covering
roughly $(\ell=160^\circ \rightarrow 220^\circ, b=-70^\circ \rightarrow
-10^\circ)$. Interesting features should produce antenna temperatures
$\sim 20$ K. This is almost 2800 square degrees and, with 2 degree
spacing, requires almost 700 profiles. See \S \ref{sampling}!

The goal here is to map the HI in the 3-d space of $(l, b, v)$ (that's
Galactic longitude, latitude, and velocity) and then---the challenging
part---present the results in one or more color images to show the
hollowed-out shell with the swept-up gas piled up at the edges.

\section{MAPPING THE NORTH POLAR SPUR and its associated EXPANDING HI SHELL}

This is a huge shell produced (and continuing) by several energetic
stellar winds and supernovae that were located in the Sco/Oph
association of stars. The shell contains not only HI but also
relativistic electrons, so the shell is visible in both the 21-cm line
and synchrotron emission. We can't easily see the latter, but we can see
the HI---and the pattern of its velocity follows that of an expanding
shell.

For this, you'd map the region covering roughly $(\ell=210^\circ
\rightarrow 20^\circ, b=0^\circ \rightarrow 90^\circ)$. (The longitude
range is $170^\circ$: $210^\circ \rightarrow 360^\circ$ plus $0^\circ
\rightarrow 20^\circ$). This is roughly 1/4 of the whole sky, so this is
a huge area---almost 10000 square degrees! (This shell has an angular
diameter of $120^\circ$). But (unfortunately) you can't measure the
equivalent 2000 HI profiles because most of this area is too far
south. At least, this makes the project do-able! In doing your
observations, you need to cover as far south as you possibly can. See
\S \ref{sampling}!

The goal here is to map the HI in the 3-d space of ${l, b, v}$ (that's
Galactic longitude, latitude, and velocity) and then---the challenging
part---present the results in one or more color images to show the
hollowed-out shell with the expanding swept-up gas piled up at the
edges.

\section{MAPPING A BIG HIGH-VELOCITY CLOUD}

When we look up, away from the Galactic plane, we see infalling
gas---some a very high velocities. It's called ``High-Velocity Gas''.
The line is weak (about 1 to 1.5 K), so you need high sensitivity---you
need to use much longer integration times than for the above projects,
at least a few minutes per point.  This project needs to map the region
bounded roughly by $(\ell=60^\circ \rightarrow 180^\circ), b=20^\circ
\rightarrow 60^\circ)$. This is about 3700 square degrees and needs
about 900 profiles at $2^\circ$ spacing. See
\S \ref{sampling}!

The goal here is to map the HI in the 3-d space of ${l, b, v}$ (that's
Galactic longitude, latitude, and velocity) and then---the challenging
part---present the results in one or more color images to show the
hollowed-out shell with the expanding swept-up gas piled up at the
edges.

\section{MAPPING THE MAGELLANIC STREAM}

The Magellanic Stream is an intergalactic tidal stream produced by
gravitational interaction between the Magellanic Clouds and the Milky
Way. The signal is weak ($\sim$ a few tenths K), so you need to use much
longer integration times than for the above projects---a minimum of 10
minutes per point. You need to map the pie-shaped area bounded roughly
by $(\ell=60^\circ \rightarrow 110^\circ), b=-90^\circ \rightarrow
-30^\circ)$. This is about 1250 square degrees and needs about 310
profiles at $2^\circ$ spacing. See \S \ref{sampling}!  Also, the LSR
velocity of this gas ranges from roughly $-400 \ {\rm to} \ -100$ km/s
along its length, so you need frequency switch by a large enough
interval!

The goal here is to map the HI in the 3-d space of ${l, b, v}$ (that's
Galactic longitude, latitude, and velocity) and then---the challenging
part---present the results in one or more color images to show the
hollowed-out shell with the expanding swept-up gas piled up at the
edges.

\section{DETECTING OH MASERS IN W49}

The OH molecule is an important one for star formation and molecular
clouds because it is one of the first to form when a diffuse atomic
cloud starts to become denser; in other words, OH marks the transition
between atomic and molecular gas. It persists at higher densities. The
ground rotational/vibrational state exhibits ``$\Lambda$ doubling'',
which produces two spectral lines; each of these is hyperfine-split
(just like the 21-cm line), and the net result is four lines centered at
1612.2310, 1665.4018, 1667.3590, 1720.5300 MHz. At sufficiently high
densities, OH can be ``pumped'' by collisions, IR, or UV radiation to
make these lines shine as masers.

The maser lines are strong, but our telescope is small, and the
strongest source, W49, will produce $< 1$ K antenna temperature. We have
interference, which makes the line difficult---maybe impossible---to
detect. We've never detected it with our equipment. 

The goal here is to detect the maser line in W49, which has $(ra,\
 dec) = (19^h 10^m 17s, +09^\circ 06.0')$. This will require
 investigating the time/frequency properties of the interference using
 plots and images; excising the interference by hand; experimenting with 
algorithms to reduce or eliminate the interference (known as ``signal
 processing''), presenting the results, and learning some of the basic
 maser theory. 

We are currently working on this project to see if it is feasible. This
project is more oriented towards computing and signal processing than
``astronomy'', and as such would not appeal to everyone.

\section{ IF YOU HAVE TIME and ARE TOTALLY IN LOVE WITH RADIO ASTRONOMY
  (AND NOTHING OR NOBODY ELSE) THEN: MEASURE THE TELESCOPE BEAMWIDTH} 

	Determine the beamwidth of the telescope by observing the Sun.  Also
answer the question: is the beam circularly symmetric? You'd think so,
given that the telescope has a circular aperture\dots

	Here you are using the Sun as a ``point source'' so that its
antenna temperature $T_A$ is proportional to the gain of the telescope as a
function of angle. The resulting $T_A$ is just the telescope's diffraction
pattern! The ``half power beam width'' (HPBW) is what radio astronomers
generally use as the single-parameter specification of the angular size of
the diffraction pattern. 

          Measure the HPBW and compare it with a theoretical estimate of
the HPBW, which you can make from the usual diffraction argument. 

%\section{COMMENTARY}

\section{BASICS OF THE 21-CM LINE.} \label{basics}

\subsection{Column Density and Mass} \label{coldensity}

	The 21-cm line, with frequency 1420.405751786 MHz, comes from
atomic hydrogen (HI).  In the terrestrial environment, H atoms quickly
become H$_2$ molecules; in interstellar space, where densities are far
lower than in the best vacuum systems on Earth and there are ``lots'' of
UV photons that dissociate H$_2$, H remains atomic unless it resides
inside dark clouds where it is shielded from starlight.

	The intensity of the 21-cm line is directly proportional to the
column density of H atoms as long as the opacity of the line is small;
this is a reasonably good approximation (but not perfect, particularly
in the Galactic plane where the line is strong). With this
approximation, 

\begin{equation} \label{nh1}
N_{HI} = 1.8 \times 10^{18} \int T_{B}(v) dv \ {\rm cm^{-2}}. 
\end{equation}

\noindent Here $N_{HI}$ is the column density of H atoms---the number of
atoms in a 1 cm$^2$ column along the line of sight; $T_{B}(v)$ is the
brightness temperature of the 21-cm line, which is a function of 
velocity $v$; and $v$ is the velocity in km s$^{-1}$. The velocity $v$
is produced by the Doppler effect, so a frequency shift $\Delta \nu$
from line center corresponds to velocity $v = - c {\Delta \nu \over
\nu}$ or, for the 21-cm line, $v = -{\Delta \nu \over 4.73 {\rm kHz}}$.
Note the minus sign! It means that positive velocities are receding---a
very convenient convention for astronomy, because of the expansion of
the Universe. 

          Note an important corollary to equation \ref{nh1}: If we
consider a small velocity interval $\Delta v$, then the number of H
atoms in the column in that velocity range is just 

\begin{equation} \label{nh2}
N_{HI}(v \rightarrow v + \Delta v) = 1.8 \times 10^{18} T_{B}(v) \Delta
v \ {\rm cm^{-2}} \ .  
\end{equation}

\noindent This is very important, because it gives us the opportunity to
make maps of the interstellar gas at different velocities and to
determine how it moves.

	To get the column density we need to get the brightness
temperature $T_B$, which is not the same as our antenna temperature
$T_A$. The relationship between these depends on the relative size of
the source and the beam. This is equation (9) in the ``Fount of All
Knowledge!'' handout, namely 

\begin{equation} \label{taeqn}
T_A \approx T_B {\Omega_s \over \Omega_s + \Omega_b} \ ,
\end{equation}

\noindent where $\Omega_s$ and $\Omega_b$ are the solid angles of the
source and beam, respectively. From this, we see that if the source
fills the beam ($\Omega_s \gtrsim \Omega_b$) then $T_A \approx \langle
T_B\rangle$ (the brackets denote the average over solid angle);
while if it is much smaller ($\Omega_s \ll \Omega_b$) then $T_A \approx
{\langle T_B \Omega_s\rangle \over \Omega_b}$, i.e.\ $T_A$ is smaller by the ratio of
the solid angles. 

	To get the mass of HI from the column density, you need to know
the area of the region in {\it linear} measure, which is just $\Omega \
d^2$, where $d$ is the distance.  Each HI atom has mass $m_H$, so the
total mass of HI in a region of size $\Omega$ is just

\begin{mathletters} \label{masseqn}
\begin{equation} 
M_{HI} = m_H \ d^2 \ \langle N_{HI} \Omega\rangle \ .
\end{equation}

\noindent Here $\langle N_{HI} \Omega \rangle$ is shorthand for the
general relation, which uses an integral:

\begin{equation}
\langle N_{HI} \Omega \rangle = \int_{\rm region} N_{HI} d \Omega
\end{equation}
\end{mathletters}

\noindent Substituting for $N_{HI}$ from equation \ref{nh2}, we have

\begin{equation}  \label{telemass}
M_{HI}(v) = 1.8 \times 10^{18} \Delta v \ d^2 \  m_H \ \langle T_B  \Omega\rangle \ {\rm
gm} \ . 
\end{equation}

\noindent Units are cgs. Sometimes there is a well-defined source with
boundaries; in this case, to get the mass of the whole source, we use
the solid angle occupied by the source, $\Omega_s$.

	Sometimes (in fact, often) we want to know the mass seen {\it by
the telescope} for some particular observation.  For an extended region
with $\Omega \gg \Omega_b$, $T_A = T_B$ and the telescope sees the solid
angle $\Omega_b$, so the product $\langle T_B \Omega\rangle = T_A
\Omega_b$ because angular area seen by the telescope is limited by its
beam size.  In contrast, if the HI emission is limited to a small
region---a small source with $\Omega_s \ll \Omega_b$---then $T_A \approx
{\langle T_B \Omega_b\rangle \over \Omega_s}$ so the product $\langle
T_B \Omega_s\rangle = T_A \Omega_b$; the telescope sees the whole
source.  

For both extremes, the product $T_B \Omega = T_A \Omega_b$. 
It's not only these two extremes, but for {\it all} sources the
product $\langle T_B \Omega \rangle= T_A \Omega_b$.  Thus, it is always true that the
mass seen by the telescope is

\begin{equation}  \label{telemass1}
M_{HI}(v) = 1.8 \times 10^{18} \Delta v \ d^2 \ m_H \ T_A  \Omega_b  \ {\rm
gm} \ . 
\end{equation}

\noindent The distance $d$ is usually a function of velocity $v$,
especially in the Galactic plane. 

	$T_A(v)$ is directly measured and $\Omega_b$ is the telescope
beam area, which is known, so it's easy to calculate $M_{HI}(v)$---but
only if you know the distance! Often you don't know the distance, but
you might have a hunch for a reasonable value for the distance. Suppose
this is 100 parsecs. In this
case, people usually evaluate the mass for this distance and, when
giving the mass, say ``The mass is $xxx \times d_{100}$, where $d_{100}$
is the distance in units of 100 pc''. 

\subsection{Converting Galactic Rotation to Doppler Velocity}

          Measurement of the Doppler shift caused by differential
Galactic rotation for $0^\circ < \ell < 90^\circ$ allows a direct
determination of the Galactic rotation curve inside the Solar circle,
using the ``tangent point'' method. See Burton's article ``Structure
of our Galaxy from Observations of HI'' in {\it Galactic and
Extragalactic Radio Astronomy, second edition} (editors: G.L. Verschuur
and K.I. Kellermann). In essence, you use the following equation.

	We won't go through the derivation here; we cover it in class,
and it is also done in Burton's article,
page 303-304.  Let $V_{Dopp}$ be the Doppler velocity (this is also the
observed LSR velocity), $V(R)$ be the Galactic rotation velocity at
Galactocentric distance $R$, and let the subscript $\odot$ denote values
at the Solar circle.  Then we have
\begin{equation} \label{vdopp}
V_{Dopp} = \left[ {V(R) \over R} - {V(R_\odot) \over R_\odot}\right]
R_\odot \sin(\ell)
\end{equation}
\noindent The values at the solar circle are $V(R_\odot) \approx 220$
km/s, $R_\odot \approx 8.5$ kpc.

\end{document}

\end



