\documentclass[11pt,preprint]{aastex}

\begin{document}

\title {WHAT CAN WE DETECT WITH THE LEUSCHNER DISH?}

	For OH, we'll try three sources. And we'll try a pulsar. One
project per group, unless you're really eager!

\section{The Best Bets for OH Sources}

	The OH sources include a set of masers associated with star
formation; a molecular cloud within which stars are forming; and the
molecular clouds towards the Galactic center, which we can see in
absorption against the strong continuum source.  The details:
\begin{enumerate}

\item The strongest OH masers are in the W3(OH) star forming region. 
The emission lies within a velocity range of about 10 km/s.  The
1665.401 MHz line is much stronger than the 1667.358 line.  There are
several lines, each of which is a little narrower than 1 kHz, so you
need good spectral resolution.  Because these lines are so narrow, you
can probably get away without an ``off'' spectrum.

	Nevertheless, it is good practice to obtain an off spectrum if
you can, and you can do this without losing source ``up'' time by taking
the off spectrum when the source is down.  Best is to choose a blank
position at the same declination but completely different right
ascension and observe over the same hour angle range that you observed
the source.  The lines should be about $10^{-3}$ of the system
temperature, so they are weak; if our dish were bigger, they'd be easy
to detect!

Position: $(ra, dec)_{2000} = (02^h27^m04.10^s, +61^\circ 52' 27.1'')$. 
For the off position, see the above paragraph.  LSR Velocity $\sim -45$
km/s.

\item The strongest molecular cloud in OH emission should be the Taurus
cloud.  This is optically thin thermal emission, visible in both lines
with an intensity ratio of about (9/5) for the (1667.358/1665.401) MHz
lines. 

Position: $(ra, dec)_{1950} = (04^h37^m, +25^\circ 45')$. This cloud is
about 1 degree in size so it doesn't have a precise position. Use an off
position that is either about 1 beamwidth away, or observe the off
spectrum at the same declination when the source is down as in (1) above.

LSR Velocity $\sim 7$ km/s; line width is about 2 km/s. The intensity
ratio is about thermal, 9:5 for 1667.358:1665.401.

\item The Galactic center

Position: Galactic coordinates $(l,b) = (0\circ, 0^\circ)$. Use an off
position that is either about 1 beamwidth in Galactic latitude (i.e.,
$(l,b) \approx (0^\circ, 5^\circ)$, or observe the off
spectrum at the same declination when the source is down.  The Galactic
center is a strong radio continuum source and the lines appear in
absorption; they should have the thermal intensity ratio.

LSR velocity: this line is very broad, covering the LSR velocity range
$\sim (-250, +150)$ km/s.  One implication is that you {\it must} use an off
spectrum; you can't reliably assume a flat filter shape over such a
large frequency range. The intensity
ratio is about thermal, 9:5 for 1667.358:1665.401.

\end{enumerate}

\section{The best bet for pulsar}

For a pulsar, the strongest one that's available to us is one of the
original four discovery pulsars, P0339+54.

Position: $(ra, dec)_{2000} =  (03^h32^m59.37^s, +54^\circ 34' 43.6'')$. 

Its nominal period is 0.7145 seconds.

\end{document} 

