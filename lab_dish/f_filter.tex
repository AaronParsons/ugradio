\documentclass[preprint]{aastex}
\usepackage{graphicx}
                                                                                
\begin{document}
\tolerance=10000
                                                                                
\def\simlt{\lower.5ex\hbox{$\; \buildrel < \over \sim \;$}}
\def\simgt{\lower.5ex\hbox{$\; \buildrel > \over \sim \;$}}
                                                                                
\title{HOW TO FOURIER FILTER YOUR REFERENCE SPECTRUM}

We haven't covered this in class, but it a fact\footnote{This is a
  result of the {\it correlation theorem} for Fourier transforms.} that when you calculate
a power spectrum using Fourier methods, as we do for our 21-cm line
detections, you can do it two ways: \begin{enumerate}

\item Measure time samples $E(t_j)$ and take the Fourier transform. For
  each frequency channel, the power spectrum is the square of the
  length of the complex vector. This is what we do.

\item Measure the autocorrelation function, which is $A(\tau) = \int
  E(t) E(t+\tau) \ dt$, and take the cosine Fourier transform of
  $A(\tau)$. (Here, $\tau$ represents time delay). 

\end{enumerate}

\noindent Here, in the second way, the important point is that---no
matter which method you use---the power
spectrum $P(f)$ is the Fourier transform of a quantity [which happens
to be $A(\tau)$]. We emphasize: $P(f)$ is the cosine Fourier transform,
{\it not} the square of the vector length as in (1). This means that we
can recover $A(\tau)$ by taking the inverse cosine transform of $P(f)$.

Now, our reference spectra are generally smooth with no fine structure
in frequency. The fine structure in frequency comes from the long time
delays in $A(\tau)$. This means that the best way to smooth our measured $P(f)$
is to \begin{enumerate}

\item Take the inverse transform of $P(f)$ to obtain $A(\tau)$.

\item Zero out $A(\tau)$
for $\tau \ge$ a suitable limit, chosen to ensure that no {\it real}
structure in $P(f)$, which is all broad in frequency, is lost; this
gets rid of the rapidly-varying noise.

\item Using
this modified version of $A(\tau)$, take its Fourier transform to
retrieve a smoothed, much less noisy version of $P(f)$.

\end{enumerate}

So here's how to do it:

\section{Derive $A(\tau)$}

The power spectrum $P(f)$ is the cosine Fourier transform of $A(\tau)$,
so $A(\tau)$ is the inverse cosine transform of $P(f)$. We could do this
using the full-fledged expression for the Fourier transform, but it's
much faster to use FFT. To force the FFT to take a cosine transform, we
construct a symmetrized version of $P(f)$. In our digital spectrum we
can write $P_n$ instead of $P(f)$; while $f$ runs from $-f_{Nyq}$ to
$+f_{Nyq}$, $n$ runs from 0 to $N-1$, where $N$ is even and is equal to
the number of spectral points in $P_n$.

Let's call the symmetrized version $P_{sym,j}$. It contains $J= 2N$
points. Consider $j$ to run from $-N$ to $N-1$. Then we have 

\begin{mathletters}
\begin{equation}
P_{sym,j} = P_{n} \ , \ for \ j=n \ and \ n=0 \to N-1
\end{equation}

\begin{equation}
P_{sym,j} = P_{n} \ , \ for \ j=-n \ and \ n=1 \to N-1
\end{equation}
\end{mathletters}

\noindent These equations define $P_{sym,j}$ for the range $j=-(N-1) \ \to
  +(N-1)$, a total of $2N-1$ points; $P_{sym}$ has an odd number of
  points, with perfect reflection symmetry about the origin.

We need to have an even number of points for the FFT, and we need to
invent the point for $j=-N$. We should not set this invented point equal
to zero; rather, we should set it equal to the nearest available point
on the original $P_n$. Thus, set $P_{sym,-N} = P_{sym, -(N-1)}$ (which is
  the same as setting $P_{sym,-N} = P_{sym, +(N-1)}$). 

\section{CALCULATE $A(\tau)$}

In IDL:

\begin{verbatim}
Psym_shift= shift( psym, N)
A = fft(Psym_shift, /inverse)
\end{verbatim}

\noindent $A$ has $2N$ elements. $\tau=0$ is the first element and
$\tau_{max}$ is the $N^{th}$ element.

\section{ZERO OUT $A(\tau)$ FOR $\tau \ge \tau_{max}$}

Choose a $\tau_{max}$. The smaller you make it, the less noisy will be the
filtered spectrum. Set $A_{tau}=0$ for $\tau \ge tau_{max}$. {\it Also,
  make sure} that you keep $A(\tau)$ symmetric by zeroing $\pm \tau$
  symmetrically. Call this
  modified version $A_{mod}$.

\section{TRANSFORM BACK TO GET THE FOURIER-FILTERED $P(f)$}

\begin{verbatim}
Pmod = fft( Amod)
\end{verbatim}

\noindent At this point, the first $N$ channels of {\tt Pmod}
are the filtered version of the original $P$. $P_{mod}$ will be complex;
the imaginary values should be zero to within machine accuracy. What you
want is the real values.

\section{BEFORE BARGING AHEAD, TRY THESE REALITY CHECKS!!!}

\begin{enumerate}

\item Carry through the steps above but {\it don't} zero any
elements of $A(\tau)$. You should regain the original spectrum!

\item After you derive $P_{mod}$, make sure the imaginary values are
  zero. If they are not, you're not zeroing $A(\tau)$ symmetrically for
  $\pm \tau$.
\end{enumerate}

\end{document}
